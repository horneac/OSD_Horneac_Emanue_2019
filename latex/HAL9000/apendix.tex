\begin{appendices}

\chapter{Reference Guide}

\section{HAL9000}

\subsection{Startup}
\label{sect:OsStart}

\projectname is a multiboot compliant OS, as a result it expects to be loaded at the physical
address specified in the multiboot header in 32-bit operating mode with paging disabled.

This section will detail the execution path starting from the assembly code which runs after the
multiboot loader gives us the control to the C code which initializes each component.

\subsubsection{Assembly code}

Code execution begins in the \func{\_\_EntryMultiboot} function defined in \file{\_mboot32.yasm}. As
stated earlier, we will not continue execution if we were not loaded by a multiboot loader.

The physical memory map describing the physical memory available in the system is retrieved using
the INT 15H E820H BIOS interrupt function. The list of potential COM ports are retrieved from the
bios data area.

After this information is acquired the OS will transition to operating in 64 bits where it will
forever remain. After this transition occurs in \func{PM32\_to\_PM64} in \file{\_transition.yasm}
the assembly code will give the control to the first C function: \func{Entry64}.

\subsubsection{C code}

\func{Entry64} starts by initializing the commonlib library(providing primitive synchronization
 mechanisms, asserts and other utility functions).

Control then reaches \func{SystemInit} which validates we are capable of executing on the CPU we are
currently running (\func{CpuMuValidateConfiguration}). Serial communication is initialized if there
was a valid COM port in bios data area.

The IDT is setup (\func{InitIdtHandlers}) and exceptions will no longer cause triple faults. Most
exceptions will not be handled but log messages will be generated which help debug the issue. In
contrast, the page fault \#PF exception is one which occurs very often and it will be handled mostly
successfully - see \cite{intelSys} Chapter 6 - Interrupt and Exception Handling and
\fullref{sect:InterruptHandling} for details.

The next things to initialize are the memory managers (\func{MmuInitSystem}):
\begin{itemize}
	\item The physical memory manager - \func{PmmInitSystem} - responsible for allocating physical
frames of memory on request. On initialization it processes the E820H memory map to determine how
much physical memory the system has and to reserve the already used memory regions.
	\item The virtual memory manager - \func{VmmInit} - responsible for allocating virtual addresses
and for mapping virtual addresses (VAs) to physical addresses (PAs). It is also responsible for
handling \#PFs.
	\item The heap - \func{\_MmuInitializeHeap} - the heap is responsible for managing large
continuous area of virtual space and for offering the possibility for other components to allocate
memory at a byte level granularity from these regions managed. 
\end{itemize}

\func{MmuInitSystem} will also create a new paging structure hierarchy and will cause a CR3 switch
to these new structures.

Next, if the multiboot loader passed any boot modules to us they will be loaded in memory by
\func{BootModulesInit}. The \file{Tests.module} file describing the tests to run is such a module.

The ACPI tables will be parsed by \func{AcpiInterfaceInit} to determine the processors present on
the system and to determine if PCI express support is available.

\func{CpuMuAllocAndInitCpu} will then be called to allocate the CPU structure for the bootstrap
processor (BSP - the processor on which the system execution starts) and to validate that the CPU
is compatible with \projectname.

This function will activate the CPU features required for operation and those for enhancing the
operating system's capabilities. Also, the main thread is created here.

\func{IomuInitSystem} is then called to initialize the I/O capabilities of the system by
initializing the following:
\begin{itemize}
	\item the IOAPIC: this is the system interrupt controller which superseded the PIC and is
responsible for interrupt delivery for legacy devices and for PCI devices which do not have
support for directly delivering interrupts through MSI or MSI-X.

	\item the IDT handlers, for the second time - first time there were no TSS stacks allocated for
the current CPU and now there are. The reason to use TSS stacks is to prevent interrupt handlers
from using an invalid or corrupt stack. More information can be found in \cite{intelSys}
Chapter 6 - Interrupt and Exception Handling.

	\item the PCI hierarchy, all PCI devices must be retrieved and placed in a device hierarchy so
proper interrupt routing can be done.

	\item the clocks: RTC - used to update the clock found in the top right-hand corner of the
display and the PIT - programmed to deliver the scheduler clock tick.

	\item the keyboard: used for receiving commands from the user operator.
\end{itemize}

\func{SmpSetupLowerMemory} will then setup the required memory structures for all the other CPUs to
start up - these are called Application Processors (APs).

\func{ProcessSystemInitSystemProcess} will then create the "System" process and attribute the only
running thread to it.

\func{ThreadSystemInitIdleForCurrentCPU} will spawn the idle thread and will enable interrupt
delivery.

The ACPI tables are parsed again through the \func{AcpiInterfaceLateInit} function for additional
information: the PCI routing tables which describe which entries of the IOAPIC are used by which
devices. This step is required for proper interrupt setup for PCI devices without MSI/MSI-x
capabilities.

All the APs will now be woken up by the \func{SmpWakeupAps} function. This function returns only
after all the processors have woken up - for more details see \fullref{sect:ApInit}. Once this
function returns the main thread of execution - which initially ran on the BSP - may be moved on
any of the other CPUs.

After the APs have all woken up all interrupts registered for devices are enabled: keyboard and
clocks.

The \func{IomuInitDrivers} function is then called to initialize each driver found in the
\var{DRIVER\_NAMES} list. These are the drivers responsible for managing the disk controller, the
abstract disk and volume concepts, the FAT32 file system and the ethernet network card.

Afterward, the \func{CmdRun} function is executed which executes all the commands received in all
the boot modules (currently only the "Tests" module) and then allows the user to issue hand-written
commands.

If the /shutdown command is given the system shuts down. To reboot the system you can use the /reset
command.

\subsubsection{AP initialization}
\label{sect:ApInit}

The APs start execution in 16 bits in the \func{TrampolineStart} function found in the
\file{\_trampoline.yasm} assembly file. The assembly code is responsible for transitioning from
real-mode in 16 bits to protected mode in 32 bits and then to long mode with paging enabled. Once
these transitions have been made the \func{ApInitCpu} C function is called.

Within this function the GDT and IDT are reloaded with their high-memory counterparts, the current
CPU structure is initialized thus starting the main thread of execution on the AP and signaling the
BSP that it has woken up. The idle thread is then initialized for the AP and the main thread exits
successfully by calling the \func{ThreadExit} function.

\subsection{Multi-core vs Single-core}
\label{sect:MultiCore}

\projectname has support  for multi-processor ssytems - what this means is that despite the fact 
that the OS starts execution on a single processor (the BSP) it detects any additional CPUs found in
the system, initializes them and begins executing code on them as well.

Due to this fact true code concurrency is achieved, i.e. code can actually be executed at the same
time on two or more processors. This cannot be said for single-core systems because only one thread
may be executing at any time on a CPU - the concurrency available on single-core systems is due to
the fact that while a thread is executing a function it may be interrupted and another thread may
be scheduled to execute in its place and it may execute the same code - but it is not executing at
the same time.

On the other hand, on multi processor systems there are as many threads executing as many processor
cores are in the system, due to this fact more threads may execute the same code at the same time.

From these statements we can derive that synchronization on single-core systems is as simple as
turning off interrupts in critical regions - thus making it impossible for the executing thread to
be preempted. While this is a bad habit for single-core systems it is useless for multi-core systems.

When interrupts are disabled, they guarantee that the current thread is not preempted, but that
doesn't stop threads executing on other processors from executing code in the same critical region.
For a further discussion on the synchronization mechanisms used in \projectname see
\fullref{sect:Synch}.

To communicate between CPUs an architectural mechanism is provided through Inter Processor
Interrupts (IPIs). Any CPU can send an interrupt to himself, any other CPU, any group of CPUs or
to all the CPUs in the system. For more information about this mechanism be sure to read \fullref{sect:IntCpuComm}.

\subsection{Why 64-bit?}
\label{sect:Why64}

You're probably curious what the advantages of 64-bit code are over 32-bit code. There are several
benefits both in terms of performance and of security.

The performance benefits are the following:
\begin{enumerate}
	\item Processor Context IDentifiers (PCIDs)

	When paging is enabled, i.e. a hardware mechanism is used to translate virtual addresses (VAs)
to physical addresses (PAs), the CPU optimizes the translation operation by caching mapping
information. Without PCIDs, on each CR3 switch, i.e. when the executing process changes, all
caching information is flushed for all the user mode mappings. The kernel mappings remain cached
because they are marked as global translations in their paging tables which tell the processor that
those mappings are available regardless of the executing process.

	When PCIDs are used the CPU will no longer flush any mappings on CR3 switches. This is because
each cached mapping is also indexed by the process's PCID which will ensure the CPU will not use
the wrong translation after a CR3 switch.

	\item Specialized system call mechanisms: SYSCALL/SYSRET

	This new mechanism available only in long mode is optimized for transitioning between user-mode
code (running at ring 3) and kernel code (running at ring 0). It is much faster than using an
interrupt gate for servicing system calls and a little faster and easier to use than the
SYSENTER/SYSEXIT mechanism.

\end{enumerate}

The security benefits are the following:
\begin{enumerate}
	\item XD (eXecution Disable)

	Pages can be marked as non-executable. This is a very important protection mechanism and is the
mechanism used for DEP (Data Execution Prevention). This is a mitigation for buffer overflow attacks
due to the fact that when used properly XD will disable execution of data found on the stack or
dynamically allocated.

	\projectname follows the W $\xor$ X principle, i.e. a memory region MUST not be both executable
and writable at the same time.

	\item Larger virtual address space (VAS)

	Because the address space is much larger when operating in long mode - \begin{math}2^{48}\end{math}
versus \begin{math}2^{32}\end{math}-  it is easier to randomize the address space layout of the
kernel and of the user application thus making the attacker's job harder to pinpoint those addresses.

	Unfortunately, in the current version, \projectname does not perform any address space layout
randomization (ASLR) and loads the kernel and user-mode applications at fixed virtual addresses.

	\item Protection Keys

	This mechanism offers an additional mechanism for controlling accesses to user-mode addresses.
If enabled, it can be used to disable read/write access at the page-level granularity for each
user-mode address. These restrictions also apply for supervisor mode accesses.

	Unfortunately, \projectname does not use this feature.

\end{enumerate}

NOTE: The XD benefit is also available when executing in 32-bit mode with PAE enabled, but the other
benefits apply exclusively to execution in long mode.

\subsection{Interrupt Handling}
\label{sect:InterruptHandling}

An interrupt notifies the CPU of some event. For our purposes, we classify interrupts into two broad
categories:
\begin{itemize}
	\item Internal interrupts - these are synchronous interrupts caused directly by CPU instructions.
Attempts at invalid memory accesses (page faults), division by 0, software interrupts and some other
activities cause internal interrupts.

	Because they are caused by CPU instructions, internal interrupts are synchronous or synchronized
with CPU instructions and cannot be disabled.

	\item External interrupts - these are asynchronous events generated outside the current CPU, i.e.
they may be generated by other CPUs, other hardware devices such as the system timer, keyboard, disk,
network controller and so on. External interrupts are asynchronous, meaning that their delivery is
not synchronized with instruction execution. Handling of external interrupts can be postponed by
disabling interrupts with \func{CpuIntrDisable} and related functions, see
\fullref{sect:DisablingInt} for details.
\end{itemize}

The CPU treats both classes of interrupts largely the same way, so \projectname has common 
infrastructure to handle both classes. The following section describes this common infrastructure.
The sections after that give the specifics of external and internal interrupts.

\subsubsection{Interrupt Infrastructure}

When an interrupt occurs, the CPU saves its most essential state on the stack and jumps to an 
interrupt handler routine. The 80x86 architecture supports 256 interrupts, numbered 0 through 255,
each with an independent handler defined in an array called the interrupt descriptor table or IDT.

In our project, \func{InitIdtHandlers} is responsible for setting up the IDT so that each entry
corresponds to a unique entry point in \file{\_isr.yasm}. The exception handlers (vectors between 0
and 31) have proper names, while the interrupt handlers are generated using a macro and have the
\func{GenericIsrN} name - where N is between 32 and 255. Because the CPU doesn’t give us any other
way to find out the interrupt number, each entry point pushes the interrupt number on the stack.
For consistent interrupt handling, a dummy error code is pushed on the stack for interrupts which
do not generate such error codes. After this information is saved the \func{PreIsrHandler} is
 called - this function saves all the general purpose registers on the stack and calls the
\func{IsrCommonHandler} C function.

\func{IsrCommonHandler} branches to \func{\_IsrExceptionHandler} for exceptions and to
\func{\_IsrInterruptHandler} for any other interrupts. These are described in the
\fullref{sect:IntInterrupt} and \fullref{sect:ExtInterrupt}.

If \func{IsrCommonHandler} handles the interrupt successfully, i.e. either the exception was
benign or the interrupt was acknowledged by a device device, control returns to \func{PreIsrHandler}.
The registers and the stack are restored and the CPU returns from the interrupt through the IRETQ
instruction.

\textbf{NOTE: Execution of both classes of interrupt handlers currently happen with interrupts
disabled.}

As a result, an interrupt handler effectively monopolizes the CPU and delays all other activities on
that CPU. Therefore, external interrupt handlers should complete as quickly as they can. Anything
that require much CPU time should instead run in a kernel thread, possibly one that the interrupt
unblocks using a synchronization primitive.

\subsubsection{Internal Interrupt Handling}
\label{sect:IntInterrupt}

Internal interrupts are caused directly by CPU instructions executed by the running kernel thread or
user process (from project 2 onward). An internal interrupt is therefore said to arise in a
process context.

In the current project implementation, the only type of exceptions \func{\_IsrExceptionHandler} is 
able to solve are page fault exceptions (see \fullref{sect:PfHandling} for more details.
For any other exceptions or when a \#PF cannot be satisfied the interrupt frame and some of the
stack area is logged to help debug the problem and the system asserts.

\subsubsection{External Interrupt Handling}
\label{sect:ExtInterrupt}

External interrupts are caused by events outside the CPU. They are asynchronous, so they can be
invoked at any time that interrupts have not been disabled. We say that an external interrupt runs
in an interrupt context.

In an external interrupt, the interrupt frame or the processor state is not passed to the handler
because it is not very meaningful. It describes the state of the thread or process that was
interrupted, but there is no way to predict which one that is. It is possible, although rarely
useful, to examine it, but modifying it is a recipe for disaster.

An external interrupt handler must \textbf{not sleep, yield or block}, which rules out using executive
synchronization mechanisms - primitive synchronization or interlocked operations can still be used,
see \fullref{sect:Synch}.
Sleeping in interrupt context would effectively put the interrupted thread to sleep and block any
interrupts of lower or equal priority than the one currently serviced. This would be disastrous
and may cause the next scheduled thread to run indefinitely because the scheduler interrupt may
never occur again.

Interrupt delivery to a CPU is controlled by the IOAPIC (system-wide) and the LAPIC (per CPU). If
the external interrupt was acknowledged by a device driver it is considered handled and it is
acknowledged (see \func{IomuAckInterrupt}).

After the interrupt was acknowledged \func{\_IsrInterruptHandler} checks if it should preempt the
current thread or not and takes the appropriate action.

To register an interrupt handler for a device \func{IoRegisterInterrupt} and
\func{IoRegisterInterruptEx} can be used.

\section{Threads}
\label{sect:Threads}

\subsection{Thread Structure}

The structure defining a thread is found in \file{thread\_internal.h}.
\begin{lstlisting}[caption={Thread structure},label={lst:ThreadStruct}]
typedef struct _THREAD
{
    REF_COUNT               RefCnt;

    TID                     Id;
    char*                   Name;

    // Currently the thread priority is not used for anything
    THREAD_PRIORITY         Priority;
    THREAD_STATE            State;

    // valid only if State == ThreadStateTerminated
    STATUS                  ExitStatus;
    EX_EVENT                TerminationEvt;

    volatile THREAD_FLAGS   Flags;

    // Lock which ensures there are no race conditions between a thread that
    // blocks and a thread on another CPU which wants to unblock it
    LOCK                    BlockLock;

    // List of all the threads in the system (including those blocked or dying)
    LIST_ENTRY              AllList;

    // List of the threads ready to run
    LIST_ENTRY              ReadyList;

    // List of the threads in the same process
    LIST_ENTRY              ProcessList;

    // Incremented on each clock tick for the running thread
    QWORD                   TickCountCompleted;

    // Counts the number of ticks the thread has currently run without being
    // de-scheduled, i.e. if the thread yields the CPU to another thread the
    // count will be reset to 0, else if the thread yields, but it will
    // scheduled again the value will be incremented.
    QWORD                   UninterruptedTicks;

    // Incremented if the thread yields the CPU before the clock
    // ticks, i.e. by yielding or by blocking
    QWORD                   TickCountEarly;

    // The highest valid address for the kernel stack (its initial value)
    PVOID                   InitialStackBase;

    // The size of the kernel stack
    DWORD                   StackSize;

    // The current kernel stack pointer (it gets updated on each thread switch,
    // its used when resuming thread execution)
    PVOID                   Stack;

    // MUST be non-NULL for all threads which belong to user-mode processes
    PVOID                   UserStack;

    struct _PROCESS*        Process;
} THREAD, *PTHREAD;
\end{lstlisting}

We will summarize the most important fields:
\begin{itemize}
	\item TID Id

	Unique identifier, begins at 0 and is incremented by \macro{TID\_INCREMENT} (currently 4) for
each thread created. There is no need for recycling the ids because a TID is defined as a QWORD and
 even if the increment is 4, we'll run out of IDs (i.e. wrap around to 0) after \begin{math}2^{62}\end{math}
(4,611,686,018,427,387,904) threads are created. We *probably* won't be running for that long.

	\item THREAD\_PRIORITY Priority

	A thread priority, ranging from \macro{ThreadPriorityLowest} (0) to 
\macro{ThreadPriorityMaximum} (31). Lower numbers correspond to lower priorities, so that priority 0
is the lowest priority and priority 31 is the highest. \projectname as provided ignores thread 
priorities, but you will implement priority scheduling in project 1.

	\item THREAD\_STATE State

	This field best describes what the thread is currently doing (running on the CPU, waiting for a
resource, waiting to be scheduled or dying). See \fullref{sect:ThreadStates} for details.

	\item LIST\_ENTRY AllList

	This list element is used to link the thread into the list of all threads. Each thread
is inserted into this list when it is created and removed when it exits. The 
\func{ThreadExecuteForEachThreadEntry} function should be used to iterate over all threads. See
\fullref{sect:Lists} to see how linked lists work in \projectname.

	\item LIST\_ENTRY ReadyList

	A list element used to put the thread into doubly linked lists, either ready list (the list of 
threads ready to run) or a list of threads waiting on an executive resource. It can do double duty
because a thread waiting on an executive resource is not ready, and vice versa.

	\item LIST\_ENTRY ProcessList

	A list element used to link all the threads in the same process. On the first project you will
see this list identical to the ready list because there is only one process: the "System" process
representing the \projectname kernel.

	\item PVOID InitialStackBase

	Useful only for debugging purposes in the first project. Will be used in the second project to
determine the user threads kernel stacks. More on this in \fullref{sect:InterruptHandling}.

	\item PVOID Stack

	Every thread has its own stack to keep track of its state. When the thread is running, the CPU’s
stack pointer register tracks the top of the stack and this member is unused. But when the CPU
switches to another thread, this member saves the thread’s stack pointer. No other members are 
needed to save the thread’s registers, because the other registers that must be saved are saved on
the stack.

	When an interrupt occurs, whether in the kernel or a user program, an
\struct{INTERRUPT\_STACK\_COMPLETE} structure is pushed onto the stack. When the interrupt occurs in
a user program, this structure is always at the address pointed by \var{InitialStackBase}.

	\item PVOID UserStack

	Will be NULL for all the threads created in the first project. This field is valid only for
threads belonging to user-mode processes and points to the stack which is used by the thread when
executing user-mode code.
\end{itemize}

\subsection{Thread Functions}

The functions implemented by \file{thread.c} are exposed in two separate files: \file{thread.h}
containing the functions which may be used by any components (drivers or exposed as system calls)
and \file{thread\_internal.h} which should only be used by the components in \projectname which
work closely with the thread module.

\begin{lstlisting}[caption={Thread Public Interface},label={lst:ThPublicFuncs}]
//******************************************************************************
// Function:     ThreadCreate
// Description:  Spawns a new thread named Name with priority Function which
//               will execute the function Function which will receive as its
//               single parameter Context. The function returns a pointer
//               (handle) to the thread structure.
// Returns:      STATUS
// Parameter:    IN_Z char * Name
// Parameter:    IN THREAD_PRIORITY Priority
// Parameter:    IN PFUNC_ThreadStart Function
// Parameter:    IN_OPT PVOID Context
// Parameter:    OUT_PTR PTHREAD * Thread
// NOTE:         The thread may terminate at any time, but its data structure
//               will not be un-allocated until the handle receive in Thread is
//               closed with ThreadCloseHandle.
//******************************************************************************
STATUS
ThreadCreate(
    IN_Z        char*               Name,
    IN          THREAD_PRIORITY     Priority,
    IN          PFUNC_ThreadStart   Function,
    IN_OPT      PVOID               Context,
    OUT_PTR     PTHREAD*            Thread
    );

//******************************************************************************
// Function:     ThreadYield
// Description:  Yields the CPU to the scheduler, which picks a new thread to
//               run. The new thread might be the current thread, so you can't
//               depend on this function to keep this thread from running for
//               any particular length of time.
// Returns:      void
// Parameter:    void
//******************************************************************************
void
ThreadYield(
    void
    );

//******************************************************************************
// Function:     ThreadExit
// Description:  Causes the current thread to exit. Never returns.
// Returns:      void
// Parameter:    IN STATUS ExitStatus
//******************************************************************************
void
ThreadExit(
    IN      STATUS              ExitStatus
    );

//******************************************************************************
// Function:     ThreadWaitForTermination
// Description:  Waits for a thread to terminate. The exit status of the thread
//               will be placed in ExitStatus.
// Returns:      void
// Parameter:    IN PTHREAD Thread
// Parameter:    OUT STATUS * ExitStatus
//******************************************************************************
void
ThreadWaitForTermination(
    IN      PTHREAD             Thread,
    OUT     STATUS*             ExitStatus
    );

//******************************************************************************
// Function:     ThreadCloseHandle
// Description:  Closes a thread handle received from ThreadCreate. This is
//               necessary for the structure to be destroyed when it is no
//               longer needed.
// Returns:      void
// Parameter:    INOUT PTHREAD Thread
// NOTE:         If you need to wait for a thread to terminate or find out its
//               termination status call this function only after you called
//               ThreadWaitForTermination.
//******************************************************************************
void
ThreadCloseHandle(
    INOUT   PTHREAD             Thread
    );

//******************************************************************************
// Function:     ThreadGetName
// Description:  Returns the thread's name.
// Returns:      const char*
// Parameter:    IN_OPT PTHREAD Thread If NULL returns the name of the
//               current thread.
//******************************************************************************
const
char*
ThreadGetName(
    IN_OPT  PTHREAD             Thread
    );

//******************************************************************************
// Function:     ThreadGetId
// Description:  Returns the thread's ID.
// Returns:      TID
// Parameter:    IN_OPT PTHREAD Thread - If NULL returns the ID of the
//               current thread.
//******************************************************************************
TID
ThreadGetId(
    IN_OPT  PTHREAD             Thread
    );

//******************************************************************************
// Function:     ThreadGetPriority
// Description:  Returns the thread's priority. In the presence of
//               priority donation, returns the higher(donated) priority.
// Returns:      THREAD_PRIORITY
// Parameter:    IN_OPT PTHREAD Thread - If NULL returns the priority of the
//               current thread.
//******************************************************************************
THREAD_PRIORITY
ThreadGetPriority(
    IN_OPT  PTHREAD             Thread
    );
\end{lstlisting}

\begin{lstlisting}[caption={Thread Private Interface},label={lst:ThPrivateFuncs}]
//******************************************************************************
// Function:     ThreadSystemPreinit
// Description:  Basic global initialization. Initializes the all threads list,
//               the ready list and all the locks protecting the global
//               structures.
// Returns:      void
// Parameter:    void
//******************************************************************************
void
_No_competing_thread_
ThreadSystemPreinit(
    void
    );

//******************************************************************************
// Function:     ThreadSystemInitMainForCurrentCPU
// Description:  Call by each CPU to initialize the main execution thread. Has a
//               different flow than any other thread creation because some of
//               the thread information already exists and it is currently
//               running.
// Returns:      STATUS
// Parameter:    void
//******************************************************************************
STATUS
ThreadSystemInitMainForCurrentCPU(
    void
    );

//******************************************************************************
// Function:     ThreadSystemInitIdleForCurrentCPU
// Description:  Called by each CPU to spawn the idle thread. Execution will not
//               continue until after the idle thread is first scheduled on the
//               CPU. This function is also responsible for enabling interrupts
//               on the processor.
// Returns:      STATUS
// Parameter:    void
//******************************************************************************
STATUS
ThreadSystemInitIdleForCurrentCPU(
    void
    );

//******************************************************************************
// Function:     ThreadCreateEx
// Description:  Same as ThreadCreate except it also takes an additional
//               parameter, the process to which the thread should belong. This
//               function must be called for creating user-mode threads.
// Returns:      STATUS
// Parameter:    IN_Z char * Name
// Parameter:    IN THREAD_PRIORITY Priority
// Parameter:    IN PFUNC_ThreadStart Function
// Parameter:    IN_OPT PVOID Context
// Parameter:    OUT_PTR PTHREAD * Thread
// Parameter:    INOUT struct _PROCESS * Process
//******************************************************************************
STATUS
ThreadCreateEx(
    IN_Z        char*               Name,
    IN          THREAD_PRIORITY     Priority,
    IN          PFUNC_ThreadStart   Function,
    IN_OPT      PVOID               Context,
    OUT_PTR     PTHREAD*            Thread,
    INOUT       struct _PROCESS*    Process
    );

//******************************************************************************
// Function:     ThreadTick
// Description:  Called by the timer interrupt at each timer tick. It keeps
//               track of thread statistics and triggers the scheduler when a
//               time slice expires.
// Returns:      void
// Parameter:    void
//******************************************************************************
void
ThreadTick(
    void
    );

//******************************************************************************
// Function:     ThreadBlock
// Description:  Transitions the running thread into the blocked state. The
//               thread will not run again until it is unblocked (ThreadUnblock)
// Returns:      void
// Parameter:    void
//******************************************************************************
void
ThreadBlock(
    void
    );

//******************************************************************************
// Function:     ThreadUnblock
// Description:  Transitions thread, which must be in the blocked state, to the
//               ready state, allowing it to resume running. This is called when
//               the resource on which the thread is waiting for becomes
//               available.
// Returns:      void
// Parameter:    IN PTHREAD Thread
//******************************************************************************
void
ThreadUnblock(
    IN      PTHREAD              Thread
    );

//******************************************************************************
// Function:     ThreadYieldOnInterrupt
// Description:  Returns TRUE if the thread must yield the CPU at the end of
//               this interrupt. FALSE otherwise.
// Returns:      BOOLEAN
// Parameter:    void
//******************************************************************************
BOOLEAN
ThreadYieldOnInterrupt(
    void
    );

//******************************************************************************
// Function:     ThreadTerminate
// Description:  Signals a thread to terminate.
// Returns:      void
// Parameter:    INOUT PTHREAD Thread
// NOTE:         This function does not cause the thread to instantly terminate,
//               if you want to wait for the thread to terminate use
//               ThreadWaitForTermination.
// NOTE:         This function should be used only in EXTREME cases because it
//               will not free the resources acquired by the thread.
//******************************************************************************
void
ThreadTerminate(
    INOUT   PTHREAD             Thread
    );

//******************************************************************************
// Function:     ThreadTakeBlockLock
// Description:  Takes the block lock for the executing thread. This is required
//               to avoid a race condition which would happen if a thread is
//               unblocked while in the process of being blocked (thus still
//               running on the CPU).
// Returns:      void
// Parameter:    void
//******************************************************************************
void
ThreadTakeBlockLock(
    void
    );

//******************************************************************************
// Function:     ThreadExecuteForEachThreadEntry
// Description:  Iterates over the all threads list and invokes Function on each
//               entry passing an additional optional Context parameter.
// Returns:      STATUS
// Parameter:    IN PFUNC_ListFunction Function
// Parameter:    IN_OPT PVOID Context
//******************************************************************************
STATUS
ThreadExecuteForEachThreadEntry(
    IN      PFUNC_ListFunction  Function,
    IN_OPT  PVOID               Context
    );

//******************************************************************************
// Function:     GetCurrentThread
// Description:  Returns the running thread.
// Returns:      void
//******************************************************************************
#define GetCurrentThread()      ((THREAD*)__readmsr(IA32_FS_BASE_MSR))

//******************************************************************************
// Function:     SetCurrentThread
// Description:  Sets the current running thread.
// Returns:      void
// Parameter:    IN PTHREAD Thread
//******************************************************************************
void
SetCurrentThread(
    IN      PTHREAD     Thread
    );

//******************************************************************************
// Function:     ThreadSetPriority
// Description:  Sets the thread's priority to new priority. If the
//               current thread no longer has the highest priority, yields.
// Returns:      void
// Parameter:    IN THREAD_PRIORITY NewPriority
//******************************************************************************
void
ThreadSetPriority(
    IN      THREAD_PRIORITY     NewPriority
    );
\end{lstlisting}

\subsection{Thread States}
\label{sect:ThreadStates}

The states through which each thread can go through are: \macro{Ready}, \macro{Running},
\macro{Blocked} and \macro{Dying}. This is illustrated in \fullref{fig:ThreadStates}.
\begin{itemize}
	\item Ready: The thread is in the global thread ready list waiting to receive CPU time. It is
	ready because it is not waiting for any resource.
	\item Running: The thread is currently executing on a CPU: it will run until one the following
	happens:
		\begin{enumerate}
			\item The thread's time quantum expires, a clock interrupt occurs, the scheduler
			chooses a different thread to run and places the current one in the ready list.
			\item The thread requires a resource to continue execution and it is not currently
			available so it will be blocked.
			\item The thread willingly yields the CPU to another thread and gets placed in the ready
			list.
		\end{enumerate}
	\item Blocked: The thread is waiting for a currently unavailable resource. Once the resource
	will become available it will be moved to the ready list.
	\item Dying: The thread has finished its execution or another thread forcefully terminated it.
\end{itemize}

\begin{figure}
	\centering
	\includegraphics{ThreadStates}
		\caption{Thread States}
	\label{fig:ThreadStates}
\end{figure}

The \macro{Initializing} and \macro{Destroyed} states are pseudo-states and have no relevance in the
code. These states are illustrated for an easier understanding on where a thread starts execution
and where it ends it.

\subsection{Thread Switching} 
\label{sect:ThreadSwitch}

\func{\_ThreadSchedule} is responsible for switching threads. It is only called by the three exposed
thread functions that need to switch threads: \func{ThreadBlock}, \func{ThreadExit} and
\func{ThreadYield}. Before any of these functions call \func{\_ThreadSchedule}, they disable
interrupts so as not to be de-scheduled by a timer while calling these already de-scheduling
functions.

The \func{\_ThreadSchedule} function does the following: it acquires the ready list lock, calls
\func{\_ThreadGetReadyThread} to remove the next thread from the ready list (or the idle thread if
there are no ready threads).

If there are no ready threads and the current thread is still capable of running it is preferred
instead of the idle thread, but if there is a ready thread prepared it will be the one scheduled
instead of the old one.

If the next thread to be executed is different than the old thread the following will happen:
\begin{itemize}
	\item If the new process differs from the old one a CR3 switch will occur and the paging tables
of the scheduled process will be activated.
	\item The \func{ThreadSwitch} assembly function is called - this function stores a pointer to 
the current stack of the old thread in the \var{Stack} member and restores the stack pointer of the
new thread.
\end{itemize}

After the thread switch occurs the \func{ThreadCleanupPostSchedule} function is called to release
the ready list lock (this cannot be done before the old thread is de-scheduled else it might be
scheduled on two processors at once) and the block lock of the previous thread is also released if
it is held by the current CPU (this cannot be released earlier either because the thread cannot be
unblocked until it was blocked and it is not blocked while it is still executing on the CPU).

This function will also dereference the previous thread in case it is dying. The thread structure
may still live on if its creator did not close its handle (by calling \func{ThreadCloseHandle}).

The reason why there is a separate \func{ThreadCleanupPostSchedule} function and these operations do
not occur in the \func{\_ThreadSchedule} function is that the latter function is not called on thread
creation while the former is. For more details see \fullref{sect:ThreadInit}.

\subsection{Thread Initialization}
\label{sect:ThreadInit}

One of the trickiest parts when implementing threads is getting them to start-up. Thread switching
is easy when you have running threads.

\projectname applies the following trick to get the first thread running: it prepares on the stack
a \struct{PROCESSOR\_STATE} structure for setting up the initial register values, the address of the
\func{ThreadStart} function as the first return address and an \struct{INTERRUPT\_STACK} structure
for simulating a return from an interrupt (this is the only way to start a user-mode thread and 
there was no use of having a different start-up method for kernel threads).

For an illustration of the initial stack frame see the ASCII figure above
\func{\_ThreadSetupInitialState} and check out the implementation for further details.

When the newly created thread starts its first execution in \func{ThreadSwitch} it will call the
\func{RestoreRegisters} function to setup its initial register values, it will return at the start
of the \func{ThreadStart} function, call \func{ThreadCleanupPostSchedule} and finally perform an
IRETQ instruction which will cause the thread to continue execution in \func{\_ThreadKernelFunction}
for kernel threads. For user-mode threads (project 2) see \fullref{sect:UserThreads}.

This function calls the PFUNC\_ThreadStart function received as parameter for \func{ThreadCreate}.
Its only responsibility is to make sure \func{ThreadExit} is called by each thread even if
not explicitly coded.

\subsubsection{User-mode Threads}
\label{sect:UserThreads}

The IRETQ instruction will cause the kernel mode thread to switch its privilege level from ring 0
to ring 3 and to start executing user-mode code. The user-mode function executed depends on which
user thread is started.

If the main thread is started (i.e. the first thread in the user-mode process) the function executed
will be determined by the AddressOfEntryPoint field defined in the Portable Executable (PE) header.
For our project, this will correspond to the \func{\_\_start} function implemented in the
UsermodeLibrary project. This function is responsible for setting up the environment for a user-mode
application before actually giving it control. Similar to \func{\_ThreadKernelFunction} this
function also makes sure that the thread exits properly through the \func{SyscallThreadExit} call.

If the main thread wishes to create additional threads it should call \func{UmCreateThread}. The
UsermodeLibrary will internally call \func{SyscallThreadCreate} with \func{\_\_start\_thread} as the
function. This is to ensure that the newly spawned thread properly exits through the
\func{SyscallThreadExit} system call.

\section{Interprocessor Communication}
\label{sect:IntCpuComm}

\projectname provides an abstraction over the x86 mechanism of sending IPIs to other CPUs. This is done through the \func{SmpSendGenericIpi} and \func{SmpSendGenericIpiEx} functions as shown in \fullref{lst:SmpIpi}. \fullref{sect:SmpIpiParam} provides an overview of the parameters required by these functions, while \fullref{sect:SmpIpiExamples} offers a few common usage examples.

\begin{lstlisting}[caption={IPI functions},label={lst:SmpIpi}]
// Calls SmpSendGenericIpiEx with SmpIpiSendToAllExcludingSelf causing the
// BroadcastFunction to be executed on each CPU except the one that is calling
// the function.
STATUS
SmpSendGenericIpi(
    IN      PFUNC_IpcProcessEvent   BroadcastFunction,
    IN_OPT  PVOID                   Context,
    IN_OPT  PFUNC_FreeFunction      FreeFunction,
    IN_OPT  PVOID                   FreeContext,
    IN      BOOLEAN                 WaitForHandling
    );

STATUS
SmpSendGenericIpiEx(
    IN      PFUNC_IpcProcessEvent   BroadcastFunction,
    IN_OPT  PVOID                   Context,
    IN_OPT  PFUNC_FreeFunction      FreeFunction,
    IN_OPT  PVOID                   FreeContext,
    IN      BOOLEAN                 WaitForHandling,
    IN _Strict_type_match_
            SMP_IPI_SEND_MODE       SendMode,
    _When_(SendMode == SmpIpiSendToCpu || SendMode == SmpIpiSendToGroup, IN)
            SMP_DESTINATION         Destination
    );
\end{lstlisting}
\subsection{Parameter overview}
\label{sect:SmpIpiParam}

Let's have a look at the parameters required for \func{SmpSendGenericIpiEx}:
\begin{enumerate}
	\item \textbf{BroadcastFunction} - This is the function which will be executed by the targeted CPUs.

	\item Context - If you want to pass some information to the targeted CPUs you can pass a pointer to this field.

	\item FreeFunction - If you have allocated your Context dynamically you can specify a function to be executed after the targets complete the BroadcastFunction - useful for cleanup.

	\item \textbf{WaitForHandling}
	\	\begin{itemize}
		\item When TRUE this function blocks until all the targeted CPUs execute the BroadcastFunction
		\item When FALSE the function returns after sending the IPI
		\end{itemize}

	\item Determines the wait in which the final parameter is interpreted:
		\begin{itemize}
			\item SmpIpiSendToSelf - Will execute the BroadcastFunction on the current CPU.
			\item \textbf{SmpIpiSendToAllExcludingSelf} - Will execute the BroadcastFunction on all the CPUs \textbf{except the current CPU}.
			\item SmpIpiSendToAllIncludingSelf - Will execute the BroadcastFunction on all the CPUs \textbf{including the current CPU}.
			\item \textbf{SmpIpiSendToCpu} - Will execute the BroadcastFunction on a specific CPU - the destination is specified in the last parameter.
			\item SmpIpiSendToGroup - Will execute the BroadcastFunction on a specific group of CPUs - the destination is specified in the last paramter.
		\end{itemize}

	\item \textbf{Destination} - Must be completed only when SmpIpiSendToCpu or SmpIpiSendToGroup are used
		\begin{itemize}
			\item SmpIpiSendToCpu: The target CPU is given by the ApicId.
			\item SmpIpiSendToGroup - The target group of CPUs are formed through a logical OR of the LogicalApicId values of the targeted processors.
		\end{itemize}
\end{enumerate}

\subsection{Usage Examples}
\label{sect:SmpIpiExamples}

In this section we provide the following examples:
\begin{itemize}
	\item \fullref{lst:IpiExclSelfNoInfo} - execute a function on all the CPUs except the current one. The issuing CPU does not wait for the targeted CPUs to complete execution before continuing.

	A possible outcome of the execution on a 4 core system could be:

	\begin{verbatim}
"Hello from CPU 0x2 [0x4]"
"Hello from CPU 0x0 [0x1]"
"Hello from CPU 0x1 [0x2]"
"Hello from CPU 0x3 [0x8]"
	\end{verbatim}

	\begin{lstlisting}[caption={All the CPUs except the current one - no information passing},label={lst:IpiExclSelfNoInfo}]
static FUNC_IpcProcessEvent _CmdIpiCmd;

status = SmpSendGenericIpi(_CmdIpiCmd, NULL, NULL, NULL, FALSE);
if (!SUCCEEDED(status))
{
    LOG_FUNC_ERROR("SmpSendGenericIpi", status);
}

LOG("Hello from CPU 0x%02x [0x%02x]\n", pCpu->ApicId, pCpu->LogicalApicId);

static
STATUS
(__cdecl _CmdIpiCmd)(
    IN_OPT  PVOID   Context
    )
{
    PCPU* pCpu;

    UNREFERENCED_PARAMETER(Context);

    pCpu = GetCurrentPcpu();

    LOG("Hello from CPU 0x%02x [0x%02x]\n", pCpu->ApicId, pCpu->LogicalApicId);

    return STATUS_SUCCESS;
}
	\end{lstlisting}

	\item \fullref{lst:IpiExclSelfWithInfo} is similar to the previous example in the sense that the function will be executed by all the CPUs except the issuing one, however this time the CPUs will complete in an array the timestamp at which they executed the function and the issuing CPU will wait for their execution to complete.

A possible outcome of the execution on a 4 core system could be:

	\begin{verbatim}
"Timestamp for CPU 0 is 2305200"
"Timestamp for CPU 1 is 2302200"
"Timestamp for CPU 2 is 2303600"
"Timestamp for CPU 3 is 2289000"
	\end{verbatim}	

	\begin{lstlisting}[caption={All the CPUs except the current one - passing information},label={lst:IpiExclSelfWithInfo}]
static FUNC_IpcProcessEvent _CmdGetTimestamps;

// determine the number of active CPUs in the system
DWORD cpuCount = SmpGetNumberOfActiveCpus();

// allocate memory to pass to the broadcast function
QWORD* timeStamps = ExAllocatePoolWithTag(PoolAllocateZeroMemory, cpuCount * sizeof(QWORD), HEAP_TEMP_TAG, 0);
ASSERT(timeStamps != NULL);

status = SmpSendGenericIpi(_CmdGetTimestamps, timeStamps, NULL, NULL, TRUE);
if (!SUCCEEDED(status))
{
    LOG_FUNC_ERROR("SmpSendGenericIpi", status);
}

// when we get here all the other CPUs will have already completed the timeStamps array
timeStamps[GetCurrentPcpu()->ApicId] = IomuGetSystemTimeUs();

for (DWORD i = 0; i < cpuCount; ++i)
{
    LOG("Timestamp for CPU %u is %U\n", i, timeStamps[i]);
}

// free the context - it is not used any more
ExFreePoolWithTag(timeStamps, HEAP_TEMP_TAG)

static
STATUS
(__cdecl _CmdGetTimestamps)(
    IN_OPT  PVOID   Context
    )
{
    PCPU* pCpu;
    QWORD* timeStamps;

    // verify that we received a valid context
    ASSERT(Context != NULL);

    pCpu = GetCurrentPcpu();

    timeStamps = Context;
    timeStamps[pCpu->ApicId] = IomuGetSystemTimeUs();

    return STATUS_SUCCESS;
}
	\end{lstlisting}

	\item \fullref{lst:IpiSendToCPU} offers an example of using the extended function (\textit{SmpSendGenericIpiEx}) by sending a function to be executed on a specific target CPU. Because the WaitForHandling parameter is TRUE we have the guarantee that by the time ew return from SmpSendGenericIpiEx the BroadcastFunction will have already been executed on the target.

	The only possible outcome of the execution on a 4 core system would be:

	\begin{verbatim}
"Hello from CPU 0x3 [0x8]"
"Hello from CPU 0x0 [0x1]"
	\end{verbatim}

\begin{lstlisting}[caption={Specific target CPU},label={lst:IpiSendToCPU}]
static FUNC_IpcProcessEvent _CmdIpiCmd;

SMP_DESTINATION destination = {0};
destination.Cpu.ApicId = 3;

status = SmpSendGenericIpiEx(_CmdIpiCmd, NULL, NULL, NULL, TRUE, SmpIpiSendToCpu, destination);
if (!SUCCEEDED(status))
{
    LOG_FUNC_ERROR("SmpSendGenericIpi", status);
}

LOG("Hello from CPU 0x%02x [0x%02x]\n", pCpu->ApicId, pCpu->LogicalApicId);

static
STATUS
(__cdecl _CmdIpiCmd)(
    IN_OPT  PVOID   Context
    )
{
    PCPU* pCpu;

    UNREFERENCED_PARAMETER(Context);

    pCpu = GetCurrentPcpu();

    LOG("Hello from CPU 0x%02x [0x%02x]\n", pCpu->ApicId, pCpu->LogicalApicId);

    return STATUS_SUCCESS;
}
	\end{lstlisting}

\end{itemize}

\section{Processes}
\label{sect:Processes}

\subsection{Process Structure}

The structure defining a process is found in \file{process\_internal.h}.

\begin{lstlisting}[caption={Process Structure},label={lst:ProcessStruct}]
typedef struct _PROCESS
{
    REF_COUNT                       RefCnt;

    // The PIDs will also be used for the CR3 PCID
    PID                             Id;

    char*                           ProcessName;

    // Command line related

    // The command line also contains the ProcessName
    char*                           FullCommandLine;
    DWORD                           NumberOfArguments;

    // Signaled when the last thread is removed from the
    // process list
    EX_EVENT                        TerminationEvt;

    // Valid only if TerminationEvt is signaled. The status
    // of the process is given by the status of the last
    // exiting thread from the process.
    STATUS                          TerminationStatus;

    MUTEX                           ThreadListLock;

    _Guarded_by_(ThreadListLock)
    LIST_ENTRY                      ThreadList;

    _Guarded_by_(ThreadListLock)
    volatile DWORD                  NumberOfThreads;

    // The difference between NumberOfThreads and ActiveThreads is the following
    // ActiveThreads represents number of threads in process which have not died have
    // NumberOfThreads includes the threads which died but have not yet been destroyed
    _Interlocked_
    volatile DWORD                  ActiveThreads;

    // Links all the processes in the global process list
    LIST_ENTRY                      NextProcess;

    // Pointer to the process' paging structures
    struct _PAGING_LOCK_DATA*       PagingData;

    // Pointer to the process' NT header information
    struct _PE_NT_HEADER_INFO*      HeaderInfo;

    // VaSpace used only for UM virtual memory allocations
    struct _VMM_RESERVATION_SPACE*  VaSpace;
} PROCESS, *PPROCESS;
\end{lstlisting}

We will summarize the most important fields:
\begin{itemize}
	\item PID Id

	Unique identifier, valid values range between 1 and 4095. These can be recycled, so while
process A has ID 7 after it dies another process may take its former ID value. The reason why PIDs
are implemented this way is because of the limitations of PCIDs.

	\item char* ProcessName

	The name of the process running - this is the name of the executable.

	\item char* FullCommandLine

	The whole process command line - including the application's name.

	\item DWORD NumberOfArguments

	The number of arguments - including the process name or put differently: the number of space
separated strings in FullCommandLine.

	\item LIST\_ENTRY ThreadList

	Linked list of threads belonging to the process.

	\item LIST\_ENTRY NextProcess

	List entry linking the global process list.
\end{itemize}

\subsection{Process functions}

The functions implemented by process.c are exposed in two separate files: \textit{process.h}
containing the functions which may be used by any components (drivers or exposed as system calls)
and \textit{process\_internal.h} which should only be used by the components in \projectname which
work closely with the process module.

\begin{lstlisting}[caption={Process Public Interface},label={lst:ProcPublic}]
//******************************************************************************
// Function:     ProcessCreate
// Description:  Creates a new process to execute the application found at
//               PathToExe with Arguments (may be NULL). The function returns a
//               pointer (handle) to the process structure.
// Returns:      STATUS
// Parameter:    IN_Z char * PathToExe
// Parameter:    IN_OPT_Z char * Arguments
// Parameter:    OUT_PTR PPROCESS * Process
// NOTE:         All the processes's threads may terminate, but the process data
//               structure will not be un-allocated until the handle receive in
//               Process is closed with ProcessCloseHandle.
//******************************************************************************
STATUS
ProcessCreate(
    IN_Z        char*       PathToExe,
    IN_OPT_Z    char*       Arguments,
    OUT_PTR     PPROCESS*   Process
    );

//******************************************************************************
// Function:     ProcessWaitForTermination
// Description:  Blocks until the process received as a parameter terminates
//               execution.
// Returns:      void
// Parameter:    IN PPROCESS Process
// Parameter:    OUT STATUS* TerminationStatus - Corresponds to the status of
//               the last exiting thread.
//******************************************************************************
void
ProcessWaitForTermination(
    IN          PPROCESS    Process,
    OUT         STATUS*     TerminationStatus
    );

//******************************************************************************
// Function:     ProcessCloseHandle
// Description:  Closes a process handle received from ProcessCreate. This is
//               necessary for the structure to be destroyed when it is no
//               longer needed.
// Returns:      void
// Parameter:    PPROCESS    Process
//******************************************************************************
void
ProcessCloseHandle(
    _Pre_valid_ _Post_invalid_
                PPROCESS    Process
    );

//******************************************************************************
// Function:     ProcessGetName
// Description:  Returns the name of the currently executing process (if the
//               parameter is NULL) or of the specified process
// Returns:      const char*
// Parameter:    IN_OPT PPROCESS Process
//******************************************************************************
const
char*
ProcessGetName(
    IN_OPT      PPROCESS    Process
    );

//******************************************************************************
// Function:     ProcessGetName
// Description:  Returns the PID of the currently executing process (if the
//               parameter is NULL) or of the specified process
// Returns:      PID
// Parameter:    IN_OPT PPROCESS Process
//******************************************************************************
PID
ProcessGetId(
    IN_OPT      PPROCESS    Process
    );

//******************************************************************************
// Function:     ProcessIsSystem
// Description:  Checks if a process or the currently executing process (if the
//               parameter is NULL) is the system process.
// Returns:      BOOLEAN
// Parameter:    IN_OPT PPROCESS Process
//******************************************************************************
BOOLEAN
ProcessIsSystem(
    IN_OPT      PPROCESS    Process
    );

//******************************************************************************
// Function:     ProcessTerminate
// Description:  Signals a process for termination (the current process will be
//               terminated if the parameter is NULL).
// Returns:      void
// Parameter:    INOUT PPROCESS Process
//******************************************************************************
void
ProcessTerminate(
    INOUT       PPROCESS    Process
    );

//******************************************************************************
// Function:     GetCurrentProcess
// Description:  Retrieves the executing process.
// Returns:      PPROCESS
// Parameter:    void
//******************************************************************************
PPROCESS
GetCurrentProcess(
    void
    );
\end{lstlisting}

\begin{lstlisting}[caption={Process Private Interface},label={lst:ProcPrivate}]
//******************************************************************************
// Function:     ProcessSystemPreinit
// Description:  Basic global initialization. Initializes the PID bitmap, the
//               process list and their associated locks.
// Returns:      void
// Parameter:    void
//******************************************************************************
_No_competing_thread_
void
ProcessSystemPreinit(
    void
    );

//******************************************************************************
// Function:     ProcessSystemInitSystemProcess
// Description:  Initializes the System process.
// Returns:      STATUS
// Parameter:    void
//******************************************************************************
_No_competing_thread_
STATUS
ProcessSystemInitSystemProcess(
    void
    );

//******************************************************************************
// Function:     ProcessRetrieveSystemProcess
// Description:  Retrieves a pointer to the system process.
// Returns:      PPROCESS
// Parameter:    void
//******************************************************************************
PPROCESS
ProcessRetrieveSystemProcess(
    void
    );

//******************************************************************************
// Function:     ProcessInsertThreadInList
// Description:  Inserts the Thread in the Process thread list.
// Returns:      void
// Parameter:    INOUT PPROCESS Process
// Parameter:    INOUT struct _THREAD * Thread
//******************************************************************************
void
ProcessInsertThreadInList(
    INOUT   PPROCESS            Process,
    INOUT   struct _THREAD*     Thread
    );

//******************************************************************************
// Function:     ProcessNotifyThreadTermination
// Description:  Called when a thread terminates execution. If this was the last
//               active thread in the process it will signal the processes's
//               termination event.
// Returns:      void
// Parameter:    IN struct _THREAD * Thread
//******************************************************************************
void
ProcessNotifyThreadTermination(
    IN      struct _THREAD*     Thread
    );

//******************************************************************************
// Function:     ProcessRemoveThreadFromList
// Description:  Removes the Thread from its container process thread list.
//               Called when a thread is destroyed.
// Returns:      void
// Parameter:    INOUT struct _THREAD * Thread
//******************************************************************************
void
ProcessRemoveThreadFromList(
    INOUT   struct _THREAD*     Thread
    );

//******************************************************************************
// Function:     ProcessExecuteForEachProcessEntry
// Description:  Iterates over the all threads list and invokes Function on each
//               entry passing an additional optional Context parameter.
// Returns:      STATUS
// Parameter:    IN PFUNC_ListFunction Function
// Parameter:    IN_OPT PVOID Context
//******************************************************************************
STATUS
ProcessExecuteForEachProcessEntry(
    IN      PFUNC_ListFunction  Function,
    IN_OPT  PVOID               Context
    );

//******************************************************************************
// Function:     ProcessActivatePagingTables
// Description:  Performs a switch to the Process paging tables.
// Returns:      void
// Parameter:    IN PPROCESS Process
// Parameter:    IN BOOLEAN InvalidateAddressSpace - if TRUE all the cached
//               translations for the Process PCID will be flushed. This option
//               is useful when a process terminates and its PCID will be
//               later used by another process.
//******************************************************************************
void
ProcessActivatePagingTables(
    IN      PPROCESS            Process,
    IN      BOOLEAN             InvalidateAddressSpace
    );
\end{lstlisting}

\subsection{Program Startup}
\label{sect:ProgramStart}

In \projectname the UsermodeLibrary library defines the entry point of every application:
\func{\_\_start} in \file{um\_lib.c}. As you can see from the code its job is to perform some basic
initialization and make sure that the \func{SyscallThreadExit} call is made after the actual
application finishes.

All user applications in \projectname must implement the \func{\_\_main} function which receives the
standard parameters for any C main thread: argc and argv.

The kernel must put the arguments for the initial function on the stack before it allows the user
program to begin executing. The arguments are passed in the same way as the normal calling
convention. See \href{https://msdn.microsoft.com/en-us/library/zthk2dkh.aspx}{MSDN Parameter Passing}.

We will now consider an example on how to handle arguments for the following command:
"SampleApplication Johnny is a good kid". First, break the commands into words: "SampleApplication",
"Johnny", "is", "a", "good" , "kid". Place the words at the top of the stack. Order doesn't matter,
because they will be referenced through pointers.

Then, push the address of each string, on the stack, in right-to-left order. These are the elements
of argv. The order ensures that argv[0] is at the lowest virtual address. The VS compiler assumes
that the stack is aligned to 16 bytes before the return address is pushed, i.e. the address of the
return address must be a multiple of 8, but not 10, i.e. its least significant nibble must be 8
(see \href{https://msdn.microsoft.com/en-us/library/ms235286.aspx}{MSDN x64 Calling Conventions}).

Then, push argv (the address of argv[0]) and argc, in that order. Finally, push a fake "return
address": although the entry function will never return, its stack frame must have the same
structure as any other.

NOTE: The C standard requires a NULL pointer sentinel to also be pushed on the stack after the last
valid string address, such that argv[argc] is NULL, however \projectname does not require this and
we will not check this in our tests.

The stack is illustrated in \fullref{fig:UserStackInit}. In this example the stack pointer would be
initialized to 0x13'FFFF'FF78.

These actions should be taken in \func{\_ThreadSetupMainThreadUserStack}, this function is called
when the main (first) thread of a process is created. However, if we think a minute, we'll see that
we have a problem: the OS has access to its virtual space and to the virtual space of the currently
running process, i.e. we cannot directly access the stack of the newly created process.

To overcome this problem we can temporarily map the physical memory described by the user virtual
pages in kernel space. For this we can use \func{MmuGetSystemVirtualAddressForUserBuffer}, as a
result we will obtain a kernel virtual mapping which we can then access. Once the stack is setup
we can free this virtual address using \func{MmuFreeSystemVirtualAddressForUserBuffer}.

\begin{figure}
	\centering
	\includegraphics[scale=0.85]{UserStackInit}
		\caption{Initial User Stack}
	\label{fig:UserStackInit}
\end{figure}

\section{Synchronization}
\label{sect:Synch}

If sharing of resources between threads is not handled in a careful, controlled fashion, the result
is usually a big mess. This is especially the case in operating system kernels, where faulty sharing
can crash the entire machine. \projectname provides several synchronization mechanisms to help out.

Depending on where and what you'll want to synchronize you have the following classes of
synchronization mechanisms:
\begin{itemize}
	\item Primitive: these mechanisms wait for a resource through busy waiting, they do not block
thread execution and do not allow for preemption because they disable interrupts before starting to
acquire the resource and leave them disabled until releasing it. These synchronization mechanisms
can be used anywhere in code. However, due to the fact that they disable interrupts, they should
be used only when synchronization interrupt handlers with other code.

	\item Executive: these mechanisms rely on the OS for management. If a thread tries to acquire
an executive resource which is unavailable the thread will be blocked and another will be scheduled
on the processor. The thread will then get unblocked once the resource is available, thus avoiding
busy waiting. This mechanism should be used to synchronize code outside of interrupt handlers.

	\item Interlocked operations: if a basic data type is shared and the operations performed on the
data are simple then atomic interlocked operations can be used. Some operations include: increment,
addition, exchange, compare and exchange. These mechanism are implemented at the hardware level.
\end{itemize}

A good way to think of the difference between primitive and executive mechanisms is the following:
the primitive mechanisms are used to synchronize CPUs, while the executive mechanisms synchronize
threads.

In both the primitive and executive cases the difference between locks and events can be thought of
in terms of ownerships. Locks have owners, the owner (either the CPU or thread) must be the one
releasing the lock, while in the case of the event anyone can signal the event or clear it.

Also, when it comes to events, both primitive and executive events are classified in two categories:
\begin{enumerate}
	\item Notification events: once an event is signaled it remains that way until it is manually
cleared. This means that if N CPUs/threads are waiting for an event they are all notified when the
event is signaled and will continue execution.

	\item Synchronization events: once an event is signaled it will remain that way only until a
CPU or thread receives the signal. This means that if N CPUs/threads are waiting for an event only
one of them will receive the notification and it will atomically clear the event not allowing any
one else to continue execution.
\end{enumerate}

\subsection{Primitive}
\label{sect:PrimSynch}

As said earlier, this class of synchronization mechanisms disable interrupts from the moment they
try to acquire the resource until they release it.

If the OS used only primitive mechanisms a tight bottle-neck would be created allowing interrupts to
come only for short periods of time, thus increasing system latency and making everything less
responsive. In this regard you should be careful and use them as little as possible.

These should be used only when synchronizing data which is shared between an interrupt handler and
other code.

\subsubsection{Locks}

\projectname supports basic spinlocks (see \fullref{lst:Spinlock}), monitor locks (see \file{monlock.h}),
read/write spinlocks (see \file{rw\_spinlock.h}) and recursive read/write spinlocks
(see \file{rec\_rw\_spinlock.h}).

There's no use in shoving the interface for every kind of lock in this document. If you're curious 
you can check out the mentioned files and read the comments to find out how they work. You should
 not use the spinlock or monitor lock functions directly, but instead you should use the interface
 exposed in \file{lock\_common.h}.

If the \func{LockInit}, \func{LockAcquire}, etc, functions will be used then the operating system
will determine dynamically which basic lock type to use: spinlocks or monitor locks (if the MONITOR
feature is supported in the CPU). Monitor locks function the same as spinlocks except they conserve
power and reduce memory contention by using a hardware mechanism to MONITOR and be notified (MWAIT)
when a memory store occurs to the monitored region.

\begin{lstlisting}[caption={Spinlock Interface},label={lst:Spinlock}]
//******************************************************************************
// Function:     SpinlockInit
// Description:  Initializes a spinlock. No other spinlock* function can be used
//               before this function is called.
// Returns:      void
// Parameter:    OUT PSPINLOCK Lock
//******************************************************************************
void
SpinlockInit(
    OUT         PSPINLOCK       Lock
    );

//******************************************************************************
// Function:     SpinlockAcquire
// Description:  Spins until the Lock is acquired. On return interrupts will be
//               disabled and IntrState will hold the previous interruptibility
//               state.
// Returns:      void
// Parameter:    INOUT PSPINLOCK Lock
// Parameter:    OUT INTR_STATE * IntrState
//******************************************************************************
void
SpinlockAcquire(
    INOUT       PSPINLOCK       Lock,
    OUT         INTR_STATE*     IntrState
    );

//******************************************************************************
// Function:     SpinlockTryAcquire
// Description:  Attempts to acquire the Lock. If it is free then the function
//               will take the lock and return with the interrupts disabled and
//               IntrState will hold the previous interruptibility state.
// Returns:      BOOLEAN - TRUE if the lock was acquired, FALSE otherwise
// Parameter:    INOUT PSPINLOCK Lock
// Parameter:    OUT INTR_STATE * IntrState
//******************************************************************************
BOOL_SUCCESS
BOOLEAN
SpinlockTryAcquire(
    INOUT       PSPINLOCK       Lock,
    OUT         INTR_STATE*     IntrState
    );

//******************************************************************************
// Function:     SpinlockIsOwner
// Description:  Checks if the current CPU is the lock owner.
// Returns:      BOOLEAN
// Parameter:    IN PSPINLOCK Lock
//******************************************************************************
BOOLEAN
SpinlockIsOwner(
    IN          PSPINLOCK       Lock
    );

//******************************************************************************
// Function:     SpinlockRelease
// Description:  Releases a previously acquired Lock. OldIntrState should hold
//               the value previous returned by SpinlockAcquire or
//               SpinlockTryAcquire.
// Returns:      void
// Parameter:    INOUT PSPINLOCK Lock
// Parameter:    IN INTR_STATE OldIntrState
//******************************************************************************
void
SpinlockRelease(
    INOUT       PSPINLOCK       Lock,
    IN          INTR_STATE      OldIntrState
    );
\end{lstlisting}

\subsubsection{Event}

These mechanisms are not used for protecting a critical region, but for notifying one or more CPUs
that an event has occurred. If the event is a synchronization event only one CPU receives the signal,
while if it's a notification event all the CPUs are informed when the signal occurs.

\begin{lstlisting}[caption={Event Interface},label={lst:Event}]
//******************************************************************************
// Function:     EvtInitialize
// Description:  Creates an EVENT object. The behavior differs depending on the
//               type of event:
//               -> EventTypeNotification: Once an event is signaled it remains
//               signaled until it is manually cleared.
//               -> EventTypeSynchronization: Once an event is signaled, the
//               first CPU which will wait detect the signal in EvtWaitForSignal
//               will also clear it, i.e. a single CPU acknowledges the event,
//               whereas the notification case all the CPUs acknowledge it.
// Returns:      STATUS
// Parameter:    OUT EVENT * Event
// Parameter:    IN EVENT_TYPE EventType
// Parameter:    IN BOOLEAN Signaled
//******************************************************************************
SAL_SUCCESS
STATUS
EvtInitialize(
    OUT     EVENT*          Event,
    IN      EVENT_TYPE      EventType,
    IN      BOOLEAN         Signaled
    );

//******************************************************************************
// Function:     EvtSignal
// Description:  Signals an event.
// Returns:      void
// Parameter:    INOUT EVENT * Event
//******************************************************************************
void
EvtSignal(
    INOUT   EVENT*          Event
    );

//******************************************************************************
// Function:     EvtClearSignal
// Description:  Clears an event signal.
// Returns:      void
// Parameter:    INOUT EVENT * Event
//******************************************************************************
void
EvtClearSignal(
    INOUT   EVENT*          Event
    );

//******************************************************************************
// Function:     EvtWaitForSignal
// Description:  Busy waits until an event is signaled.
// Returns:      void
// Parameter:    INOUT EVENT * Event
//******************************************************************************
void
EvtWaitForSignal(
    INOUT   EVENT*          Event
    );

//******************************************************************************
// Function:     EvtIsSignaled
// Description:  Checks if an event is signaled and returns instantly.
// Returns:      BOOLEAN - TRUE if the event was signaled, FALSE otherwise.
// Parameter:    INOUT EVENT * Event
//******************************************************************************
BOOLEAN
EvtIsSignaled(
    INOUT   EVENT*          Event
    );
\end{lstlisting}

\subsection{Executive}
\label{sect:ExSynch}

These synchronization mechanisms are aware of the operating system and are managed more
efficiently due to this.

These mechanisms differ from the primitive ones due to the fact that they block thread execution and
remain that way until the resource is freed and they are unblocked.

These mechanisms rely on the primitive ones for their implementation, but where the primitive
mechanisms disable interrupts for the whole duration, these require interrupts disabled only a
small amount of time in their acquire and release function; however once an executive resource has
been acquired interrupts remain enabled.

\subsubsection{Mutex}
\label{sect:Mutex}

Depending on their initialization, mutexes may either be recursive or not. If a mutex is not
recursive then the same thread is not allowed to take the mutex more than once before releasing it.

If the mutex is recursive the same thread can take the mutex as many times as it wants (up to 255
times) but it must also release it the same number of times it has acquired it.

\begin{lstlisting}[caption={Mutex Functions},label={lst:MutexFunc}]
//******************************************************************************
// Function:     MutexInit
// Description:  Initializes a mutex.
// Returns:      void
// Parameter:    OUT PMUTEX Mutex
// Parameter:    IN BOOLEAN Recursive - if TRUE the mutex may be acquired
//               several times by the same thread, else only once.
// NOTE:         A recursive mutex must be released as many times as it has been
//               acquired.
//******************************************************************************
_No_competing_thread_
void
MutexInit(
    OUT         PMUTEX      Mutex,
    IN          BOOLEAN     Recursive
    );

//******************************************************************************
// Function:     MutexAcquire
// Description:  Acquires a mutex. If the mutex is currently held the thread
//               is placed in a waiting list and its execution is blocked.
// Returns:      void
// Parameter:    INOUT PMUTEX Mutex
//******************************************************************************
ACQUIRES_EXCL_AND_REENTRANT_LOCK(*Mutex)
REQUIRES_NOT_HELD_LOCK(*Mutex)
void
MutexAcquire(
    INOUT       PMUTEX      Mutex
    );

//******************************************************************************
// Function:     MutexRelease
// Description:  Releases a mutex. If there is a thread on the waiting list it
//               will be unblocked and placed as the lock's holder - this will
//               ensure fairness.
// Returns:      void
// Parameter:    INOUT PMUTEX Mutex
//******************************************************************************
RELEASES_EXCL_AND_REENTRANT_LOCK(*Mutex)
REQUIRES_EXCL_LOCK(*Mutex)
void
MutexRelease(
    INOUT       PMUTEX      Mutex
    );
\end{lstlisting}

\subsubsection{Executive Event}
\label{sect:ExEvent}


\begin{lstlisting}[caption={Executive Event Functions},label={lst:ExEventFunc}]
//******************************************************************************
// Function:     ExEventInit
// Description:  Initializes an executive event. As in the case of primitive
//               events, these may be notification or synchronization events.
// Returns:      STATUS
// Parameter:    OUT EX_EVENT * Event
// Parameter:    IN EX_EVT_TYPE EventType
// Parameter:    IN BOOLEAN Signaled
//******************************************************************************
STATUS
ExEventInit(
    OUT     EX_EVENT*     Event,
    IN      EX_EVT_TYPE   EventType,
    IN      BOOLEAN       Signaled
    );

//******************************************************************************
// Function:     ExEventSignal
// Description:  Signals an event. If the waiting list is not empty it will
//               wakeup one or multiple threads depending on the event type.
// Returns:      void
// Parameter:    INOUT EX_EVENT * Event
//******************************************************************************
void
ExEventSignal(
    INOUT   EX_EVENT*      Event
    );

//******************************************************************************
// Function:     ExEventClearSignal
// Description:  Clears an event signal.
// Returns:      void
// Parameter:    INOUT EX_EVENT * Event
//******************************************************************************
void
ExEventClearSignal(
    INOUT   EX_EVENT*      Event
    );

//******************************************************************************
// Function:     ExEventWaitForSignal
// Description:  Waits for an event to be signaled. If the event is signaled it
//               will place the thread in a waiting list and block its
//               execution.
// Returns:      void
// Parameter:    INOUT EX_EVENT * Event
//******************************************************************************
void
ExEventWaitForSignal(
    INOUT   EX_EVENT*      Event
    );
\end{lstlisting}

\subsubsection{Executive Timer}
\label{sect:ExTimer}

\begin{lstlisting}[caption={Timer functions},label={lst:TimerFunc}]
//******************************************************************************
// Function:     ExTimerInit
// Description:  Initializes a timer to trigger to trigger at a specified time.
//               Once this function returns, the timer can be waited by multiple
//               threads at once. When the timer triggers - no matter its type -
//               all the threads waiting on it must be woken up.
// Returns:      STATUS
// Parameter:    OUT PEX_TIMER Timer - Timer to initialize
// Parameter:    IN EX_TIMER_TYPE Type - Defines the periodicity of the timer
//               (one-shot or periodic) and defines the meaning of the 3rd
//               parameter (if absolute or relative to the current time).
// Parameter:    IN QWORD TimeUs
//
// Examples:
// Initialize a one-shot timer to trigger after 1 second:
// ExTimerInit(&timer, ExTimerTypeRelativeOnce, 1 * SEC_IN_US);
//
// Initialize a periodic timer to trigger every minute:
// ExTimerInit(&timer, ExTimerTypeRleativePeriodic, 60 * SEC_IN_US);
//
// Initialize an absolute timer after the OS has run 2 minutes:
// ExTimerInit(&timer, ExTimerTypeAbsolute, 120 * SEC_IN_US);
//******************************************************************************
STATUS
ExTimerInit(
    OUT     PEX_TIMER       Timer,
    IN      EX_TIMER_TYPE   Type,
    IN      QWORD           TimeUs
    );

//******************************************************************************
// Function:     ExTimerStart
// Description:  Starts the timer countdown. If the time has already elapsed all
//               the waiting threads must be woken up.
// Returns:      void
// Parameter:    IN PEX_TIMER Timer
//******************************************************************************
void
ExTimerStart(
    IN      PEX_TIMER       Timer
    );

//******************************************************************************
// Function:     ExTimerStop
// Description:  Stops the timer countdown. All the threads waiting must be
//               woken up.
// Returns:      void
// Parameter:    IN PEX_TIMER Timer
//******************************************************************************
void
ExTimerStop(
    IN      PEX_TIMER       Timer
    );

//******************************************************************************
// Function:     ExTimerWait
// Description:  Called by a thread to wait for the timer to trigger. If the
//               timer already triggered and it's not periodic or if the timer
//               is uninitialized this function must return instantly.
// Returns:      void
// Parameter:    INOUT PEX_TIMER Timer
//******************************************************************************
void
ExTimerWait(
    INOUT   PEX_TIMER       Timer
    );

//******************************************************************************
// Function:     ExTimerUninit
// Description:  Uninitialized a timer. It may not be used in the future without
//               calling the ExTimerInit function. All threads waiting for the
//               timer must be woken up.
// Returns:      void
// Parameter:    INOUT PEX_TIMER Timer
//******************************************************************************
void
ExTimerUninit(
    INOUT   PEX_TIMER       Timer
    );

//******************************************************************************
// Function:     ExTimerCompareTimers
// Description:  Utility function to compare to two timers.
// Returns:      INT64 - if NEGATIVE => the first timers trigger time is earlier
//                     - if 0 => the timers trigger time is equal
//                     - if POSITIVE => the first timers trigger time is later
// Parameter:    IN PEX_TIMER FirstElem
// Parameter:    IN PEX_TIMER SecondElem
//******************************************************************************
INT64
ExTimerCompareTimers(
    IN      PEX_TIMER     FirstElem,
    IN      PEX_TIMER     SecondElem
    );
\end{lstlisting}

\subsection{Interlocked Operations}

If you do a quick search in the project you will see there are over 50 calls to \_Interlocked
functions. These functions are very useful when access to only a basic primitive type must be
synchronized.

One example would be \func{\_IomuUpdateSystemTime}: the scheduler clock tick occurs on each CPU,
thus each one will want to increment the system uptime with a few microseconds. There is no reason
for using a lock and causing unnecessary busy waiting when the hardware architecture can guarantee
that we can atomically add a value to a memory address.

\subsection{Disabling Interrupts}
\label{sect:DisablingInt}

\textbf{You should never have to use this mechanism for your project.} The technique and usefulness
is described here only for completeness.

Interrupts are disabled by primitive synchronization mechanisms, because their execution cannot be
preempted because they are responsible of synchronizing different CPUs and once a primitive lock is
taken on a CPU nothing else should be scheduled on that core until the lock is released.

Another region where interrupts are disabled is during a thread switch, if a thread was interrupted
while it was already in the process of yielding the CPU to another thread its state will become
inconsistent and the whole system will surely crash.

\section{Memory Management}
\label{sect:MemManagement}

One of the main responsibilities of an operating system is resource management. Well, one of the
most important resources is the system's memory: either physical memory (RAM or device memory) or
virtual memory.

You should be familiar with the concepts of virtual memory and physical memory. If you're not sure
what the difference is between these memory types you really should review a basic Operating Systems
course and come back after.

To put it very shortly, the physical memory (as the name suggests) is physical memory which either
belongs to the RAM memories in the system or to other peripheral devices connected to the system
(network cards, hard-disk controllers, USB controllers and so on).

However, \projectname and other operating systems do not work directly with physical memory, these
benefit from a mechanism implemented on the CPUs to offer a virtual address space (VAS) to software
running on the system.

Once the OS has setup the paging structures each time the CPU performs a memory access the address
used to access this memory is a virtual address (VA). The paging structures will tell the CPU
where the VA points to in physical memory and what are the access rights for which the memory can
be accessed. If the VA is not mapped to a physical address (PA) or if the access rights are
insufficient (i.e. a user-mode application wants to access a kernel-mode address or a kernel-mode
application wants to write a read-only page) a page fault (\#PF) occurs.

For more information on how \projectname handles \#PFs see \fullref{sect:PfHandling}, for more
information on how paging works you can read \cite{intelSys} Chapter 4 - Paging.

The following sections go in more detail on how physical, virtual and heap memory is managed.

\subsection{Physical Memory Management}

The physical memory manager (PMM) is responsible for managing the system's physical memory. This
means determining upon system initialization what memory is available, marking memory as reserved
when used and releasing it when it is no longer needed.

All this functionality is exposed in \file{pmm.h}. Using the INT15H E820H memory map retrieved on
system initialization the extents of physical memory is determined. This memory map is actually an
array of ranges of memory found in the system - some of it being available while other being already
reserved by the firmware or hardware devices.

The PMM tracks the availability of physical frames in the \var{m\_pmmData.AllocationBitmap} bitmap.
The size of the bitmap depends on the highest physical address available. Physical memory may 
contain gaps so we cannot use the size of available memory to determine the size of the bitmap.
Upon initialization all physical memory from 0 to the highest available is marked as reserved after
which the INT15H E820H memory map is traversed and as available memory is found it is 'released'.

Once the PMM finishes its initialization the OS can request memory using \func{PmmReserveMemory} or
\func{PmmReserveMemoryEx} and free it using \func{PmmReleaseMemory}. You probably will \textbf{NOT}
have to call these functions directly in any of your projects, we will see in the following sections
how these functions are used.

\subsection{Virtual Memory Management}
\label{sect:VMM}

The virtual memory manager (VMM) interface is exposed in \file{vmm.h}. This is the component
responsible for managing the VAS of each process and for performing the VA to PA translations by
working with the CPU's paging structures.

Each VAS is managed through a VMM\_RESERVATION\_SPACE structure. Thus, each process has its own
reservation space - pointed to by the \var{VaSpace} member of the PROCESS structure.

When memory is allocated through the \func{VmmAllocRegion} or \func{VmmAllocRegionEx} functions a
reservation for this memory is created in the VAS received as parameter or, if NULL, in the VAS of
the current process.

Each reservation is described by a VMM\_RESERVATION structure. This specifies the VA range reserved
through the \var{StartVa} and \var{Size} fields, the access rights with which the memory was
reserved, if the memory is uncacheable and if the memory is backed by a file it contains a pointer
to a file object. The reservation also maintains a bitmap which describes which of the pages reserved
are actually committed.

You may ask what's the difference between the two: when a VA range is reserved that range may no
longer be reserved by any other application. This is useful, for example, when you want to make sure
you have a continuous region of virtual space, but you don't want to use it yet.

However, when you want to start using the memory you reserved, you need to commit it - this can be
done at a page level granularity, as an example you may reserve 4 pages of memory, but you may want 
to commit only the first one and the last one.

When memory is committed it is NOT also mapped to a physical address. This is because the VMM is
lazy - see \fullref{sect:LazyMapping} for details. However, if you want to make sure that once you 
commit a VA range it is mapped you can specify the \macro{VMM\_ALLOC\_TYPE\_NOT\_LAZY} flag
to the allocate function.

To release previously allocated memory call \func{VmmFreeRegion} or \func{VmmFreeRegionEx}. De-commiting
memory can be done at the page level granularity, however releasing memory (i.e. freeing the 
reservation completely) is an all or nothing operation: either you don't release the memory, either
you release it all.

NOTE: The VMM also provides functions for mapping and un-mapping memory, however you should not use
these and use the MMU provided functions instead, see \fullref{sect:MMU}.

The functions which effectively work with the CPU paging structures to setup VA to PA translation
and to remove them are \func{VmmMapMemoryInternal} and \func{VmmUnmapMemoryEx}.

\subsubsection{Lazy Mapping}
\label{sect:LazyMapping}

The way in which mappings from virtual to physical addresses are created can be either eager or lazy.
In eager mapping, once the virtual address is allocated it's also mapped to a physical address. In
contrast, when using lazy mapping, the VA may be allocated, but it will not be mapped to a physical
address until it is actually needed, i.e. on first access to that region.

Because the lazy mapping approach is much faster than the eager one it is commonly used in modern
operating systems. Also, it is more practical not to reserve physical frames for virtual memory
allocated because most of the time most of the memory space will not be used. For example
\projectname allocates 5 GB of virtual memory for each user-mode process for their VAS management
structures, but most processes will not use more than a few KB - some even less.

Additionally, eager mapping may be impossible when allocating large ranges of virtual
addresses, as an example we may want to allocate a VA range of 1TB - however most systems on which
the OS will run certainly do not have 1TB of physical space - thus making eager mapping impossible
without supporting swap space.

In case lazy mapping is used and the 1TB range is allocated the physical frames are allocated only
to the VAs actually accessed on a per-page granularity, i.e. if from the 1TB range we access only
the first 100 bytes and the last 5 bytes we'll only have allocated 2 physical frames of memory.

\projectname's default behavior is to lazy map virtual addresses, however if you need to make sure
that once you allocated a VA range it is also backed by physical memory you can set the
\macro{VMM\_ALLOC\_TYPE\_NOT\_LAZY} flag when allocating virtual memory.

Okay, so how does the OS know that a virtual address is accessed for the first time and it needs to
be assigned to a physical frame? Well, a \#PF will occur because the VA is not mapped and once the
physical frame is reserved and the paging structures are updated to hold the VA to PA mapping
execution of the faulting instruction is restarted. For more information about handling page faults,
see \fullref{sect:PfHandling}.

\subsection{Heap Management}

A heap is a data structure for dynamically allocating memory. A heap is initialized by calling
\func{HeapInitializeSystem} - this function uses the VMM exposed function \func{VmmAllocRegion} to
reserve and commit a continuous region of virtual memory (remember - because of the lazy nature of
the VMM this does not allocate physical frames only when that memory is actually accessed).

Heap memory can then be allocated through calls to \func{HeapAllocatePoolWithTag} and freed by calls
to \func{HeapFreePoolWithTag}. The heap is the entity responsible for managing the continous region
of virtual memory.

The reason why we need a heap and we don't just use the VMM allocation functions is because most of
the time we won't need to allocate regions larger than a page size (VMM functions work at page size
granularity) and we'd waste a lot of space for nothing: while the VAS is large and this wouldn't be
a problem a physical frame will still be allocated for every allocation even if it is just one byte
in size.

Both the allocation and de-allocation functions require a DWORD Tag as a parameter - this is used
only for extra-validation: when an allocation is done the heap allocation is marked with the
requested tag, when memory is de-allocated the same tag must be used else an ASSERT will trigger.
This reduces the probability of releasing the inappropriate memory due to code errors and is also a
helpful debugging feature for tracking memory leaks, i.e. it is easier to pinpoint the component
causing the leak.

Another parameter received by the allocation function is the alignment: you may sometimes need to
allocate memory which must follow a certain alignment rule.

NOTE: For managing heap memory you should use the \func{ExAllocatePoolWithTag} and 
\func{ExFreePoolWithTag} functions exposed in \file{ex.h}.

\subsection{Memory Management Unit}
\label{sect:MMU}

The memory management unit (MMU) is responsible for initializing and coordinating the PMM, VMM and
the heaps. This unit is responsible for creating the paging structures which will be used by the
system process, for performing the switch to these structures and for mapping and reserving the
whole kernel memory.

\subsubsection{Interface}
The MMU is also responsible for aggregating the functionality exposed by the PMM, VMM and heap for
easier use and convenience. Here is a summary of what the MMU provides (see \file{mmu.h}):
\begin{itemize}
	\item Support for performing PA to VA translations through the \func{MmuMapMemoryEx} and
\func{MmuUnmapMemoryEx} functions.

	\item Support for retrieving the VA to PA translation either in the context of the current
process or by using different paging structures: \func{MmuGetPhysicalAddress} and the Ex variant.

	\item Support for validating if a buffer is valid and the process has the required access rights
for the operation desired: \func{MmuIsBufferValid}.

	\item Support for mapping a user-mode VA to a kernel-mode VA through the 
\func{MmuGetSystemVirtualAddressForUserBuffer} function.

	\item Support for loading an executable in memory and mapping it in the context of a process:
\func{MmuLoadPe}.
\end{itemize}

NOTE: For managing heap memory you should use the \func{ExAllocatePoolWithTag} and 
\func{ExFreePoolWithTag} functions exposed in \file{ex.h}.

\subsection{Page-fault handling}
\label{sect:PfHandling}

When a page fault occurs the internal exception handler will call \func{MmuSolvePageFault} passing
the faulting address (read from CR2) and the error code (pushed on the stack by the CPU) as
parameters.

This function in turns determines the access rights requested on the faulting address from the error
code, determines the paging structures which must be used (the global kernel ones or the per user
ones) from the access type (user-mode or kernel mode) and calls \func{VmmSolvePageFault}.

This later function checks to see if user-mode access was requested on a kernel page or if a kernel
mode access was requested on a user page and if any of these hold true the function fails to solve
the \#PF.

The next step is to check if the faulting address has a reservation allocated and if the memory
accessed is committed, this is done by calling \func{VmReservationCanAddressBeAccessed}. If these
checks pass then the VA should be mapped to a PA so \projectname will reserve a frame of
physical memory through \func{PmmReserveMemory}.

The next step is to actually map the VA to the newly acquired PA through \func{MmuMapMemoryInternal}.
If this reservation is backed by a file then the contents of the file from the proper offset is read
in memory by calling \func{IoReadFile}.

If the file was not backed by a file, or the contents of the file did not occupy a whole page the
remaining memory is set to 0.

If all these steps happen successfully then the exception handler will return and without any
modification on any of the processor's state before the exception the instruction which caused the
\#PF will be re-executed and the memory access will now complete successfully.

\section{Virtual Addresses}
\label{sect:VirtAddr}

A 64-bit virtual addresses can be divided into 6 sections as illustrated in \fullref{fig:VaToPa}.

\begin{verbatim}
63             48 47     39 38     30 29     21 20     12 11          0
+----------------+---------+---------+---------+---------+------------+
| Unused         |  PML4   | Dir Ptr |   Dir   |  Table  |   Offset   |
+----------------+---------+---------+---------+---------+------------+
                             Virtual Address
\end{verbatim}

Because of the way 64-bit mode works, accesses to virtual addresses require the 47th bit to be
reflected in bits 63:48, as a result these are useless for determining the table to be used for
address translation.

The next 4 pairs of 9 bits (PML4, Dir Ptr, Dir, Table) each give us an index inside a table which
takes us to an entry which holds the physical address of the next table, or in the case of the last
9 bits it gives us the final physical address where the virtual address is mapped.

The final 12 bits give us the offset inside the physical frame.

\begin{figure}
	\centering
	\includegraphics[scale=0.85]{LinearToPhysical}
		\caption{Linear-Address Translation - \cite{intelSys}}
	\label{fig:VaToPa}
\end{figure}

\projectname offers the following macros and functions for working with virtual addresses:

\begin{itemize}
	\item \func{VmmGetPhysicalAddressEx}: Retrieves the physical address for a virtual address. In
case the virtual address is not mapped the returned value is NULL. This function can also be used
to retrieve and reset the accessed and dirty bits, for details on this see \fullref{sect:ADBits}.

	\item \func{MmuMapMemoryEx}: maps a physical address range into the virtual space of a process 
or only into the virtual space of the kernel. The caller can specify the access rights requested for
the virtual address and the cacheability. The function returns the virtual mapping allocated for the
requested physical address.

	\item \func{MmuUnmapMemoryEx}: unmaps a virtual address from the address space of a process or
of the kernel.

	\item \func{MmuIsBufferValid}: checks if a virtual address range can be accessed from the
address space of a process with certain access rights.

	\item \func{\_VmIsKernelRange}: heuristically checks if an address looks like a kernel one, i.e.
it checks if bit 47 is set.

	\item \func{MmuGetSystemVirtualAddressForUserBuffer}: maps a virtual range of addresses from the
virtual address space of the specified process to a kernel virtual address range. The kernel access
rights can also be specified.

	\item \func{MmuFreeSystemVirtualAddressForUserBuffer}: frees a previously mapped user virtual
range.

	\item \var{gVirtualToPhysicalOffset}: the first valid kernel virtual address.

	\item \macro{PAGE\_SIZE}: defines the size of a page, 4 KiB.

\end{itemize}

The x86 doesn't provide any way to directly access memory given a physical address. This ability is
often necessary in an operating system kernel, for example, when working with the actual paging
structures. \projectname works around it by mapping certain regions of kernel virtual memory
one-to-one to physical memory. That is, in the case of these regions of memory the virtual address
\var{gVirtualToPhysicalOffset} accesses physical address 0, virtual address \var{gVirtualToPhysicalOffset}
+ 0x1234 accesses physical address 0x1234, and so on up. Thus, adding \var{gVirtualToPhysicalOffset}
to a physical address obtains a kernel virtual address that accesses that address; conversely,
subtracting \var{gVirtualToPhysicalOffset} from a kernel virtual address obtains the corresponding
physical address. Header \file{HAL9000.h} provides a pair of macros to do these translations:
\begin{itemize}
	\item \macro{VA2PA}: Returns the virtual address corresponding to a physical address.
	\item \macro{PA2VA}: Returns the physical address corresponding to a virtual address.
\end{itemize}

\textbf{NOTE: do not use these macros to perform the translations for any kind of addresses. As
previously mentioned these are only valid for certain memory regions, for the whole 'list', see the
ASCII diagram above \func{MmuInitSystem}.}

\section{Paging Tables}
\label{sect:PageTables}

There is a lot to talk about paging, however the Intel manuals does this better than we could ever
describe them here, so have a look over \cite{intelSys} sections 4.5 "IA-32e Paging", 4.6 "Access
Rights" and 4.8 "Accessed and Dirty Bits". This section only provides a very brief summary of the
interface provided by \projectname to interact with the paging structures.

\subsection{Creation, Destruction and Activation}

These functions create, destroy, and activate page tables. The base \projectname code already calls
these functions where necessary, so it should not be necessary to call them yourself. However, we
advise you to have a look at where they're used:

\begin{itemize}
	\item \func{\_MmuCreatePagingTables}: Creates a new base table (PML4), each process has a its
own such table.  The new page table contains \projectname's normal kernel virtual page mappings,
but no user virtual mappings.

	\item \func{\_MmuDestroyPagingTables}: Destroys a paging structure freeing all its used user
memory.

	\item \func{MmuChangeProcessSpace}: Switches the currently used paging structures for a new
set of paging structures. Called on process switches.
\end{itemize}


\subsection{Inspection and Updates}

These functions examine or update the mappings from pages to frames encapsulated by a page table.
They work on both active and inactive page tables (that is, those for running and suspended
processes), flushing the TLB as necessary.

Again, these functions are already used by \projectname and provided at a higher level interface so
you do not need to call them directly, however, you should have a look at them.

\begin{itemize}
	\item \func{PteMap}: Creates an entry in a page table to a physical address. The caller can also
specify the access rights and privilege level.

	\item \func{PteUnmap}: Zeroes a page entry thus marking it as not present.

	\item \func{PteIsPresent}: Checks if a page is present or not.

	\item \func{PteGetPhysicalAddress}: Returns the physical address mapped by the paging table
entry.
\end{itemize}

All of these functions work indifferently of the hierarchy level of the paging structure.

\subsection{Accessed and Dirty Bits}
\label{sect:ADBits}

x86 hardware provides some assistance for implementing page replacement algorithms, through a pair
of bits in the page table entry (PTE) for each page. On any read or write to a page, the CPU sets
the accessed bit to 1 in the page’s PTE, and on any write, the CPU sets the dirty bit to 1. The CPU
never resets these bits to 0, but the OS may do so.

You need to be aware of aliases, that is, two (or more) pages that refer to the same frame. When an
aliased frame is accessed, the accessed and dirty bits are updated in only one page table entry (the
one for the page used for access). The accessed and dirty bits for the other aliases are not updated.

When you implement page sharing you must manage these aliases somehow. For example, your code could
check and update the accessed and dirty bits for both addresses.

Using \func{VmmGetPhysicalAddressEx} both the accessed and dirty bits can be retrieved for a virtual
address from its corresponding page table entry. This is done by setting the corresponding output
pointer to a non-NULL address where to place the value. This function can be found in \file{vmm.h}.
\textbf{NOTE: When retrieving the value of the accessed or dirty bit the corresponding bit is
cleared from the paging table by \projectname.}

\section{List Structures}
\label{sect:Lists}

Doubly linked lists are ubiquitous in \projectname and you should learn how to use them before
working on the project. These lists are represented through LIST\_ENTRY structures. Their interface
is exposed in \file{list.h}. You do not need to learn how they work internally, in this section
we'll describe the functions exposed to work with lists. However, if you're curious on the
implementation  you can go to the header file and read the comments at the start of the file.

For those familiar with Microsoft's LIST\_ENTRY implementation this section can be skipped because
\projectname's implementation is almost identical.

Each structure that is a potential list element must embed a LIST\_ENTRY member. All of the list
functions operate on these LIST\_ENTRY members. The CONTAINING\_RECORD macro allows conversion from
a LIST\_ENTRY back to a structure object that contains it.

The head of the list is also represented through a LIST\_ENTRY, however this must not be embedded in
any structure.

An example on how to perform basic operations on a list is shown in \fullref{lst:ListExample} and is
described next.

Each structure can be in as many lists at once as LIST\_ENTRY members it has, the THREAD structure
has 3 such list entries because it is at the same time in the all threads list, in the ready/block
list and in the process list. In our example MY\_DATA has a single ListEntry thus can belong to
only one list at once.

On line 15 the list head is initialized, this must be done only ONCE and must be done BEFORE any
entry is inserted into the list.

On line 21 a new element is inserted at the back of the list, \func{InsertHeadList} is available for
insertion in the front of the list and \func{InsertOrderedList} for inserting the element in the
list following a certain order.

On line 23 we check to see if the list is empty or not.

On line 26 an element is removed from the head of the list, if we want to remove an element from its
back we can use \func{RemoveTailList}. NOTE: it wasn't necessary to check if the list is empty
before removing an element from it, if we want to remove an element from an empty list we'll
receive a pointer to the head of the list which can be easily validated.

A search in the list can be done using \func{ListSearchForElement} and for executing a function for
each list element \func{ForEachElementExecute} can be used.

Real usage examples can be found throughout the \projectname's code, especially in \file{thread.c},
\file{process.c}, \file{heap.c}, \file{iomu.c}.

\begin{lstlisting}[caption={List Usage Example},label={lst:ListExample},numbers=left]
// list head
LIST_ENTRY gGlobalList;

typedef struct _MY_DATA
{
	...
	// MY_DATA elements will link in the global list through the ListEntry field
	LIST_ENTRY		ListEntry;
	...
	BYTE			Data;
} MY_DATA, *PMY_DATA;

void SomeFunction(void)
{
	InitializeListHead(&gGlobalList);

	PMY_DATA pData = ExAllocatePoolWithTag(0, sizeof(MY_DATA), HEAP_TEST_TAG, 0);
	ASSERT(pData != NULL);

	// Inserts the data element at the end of the list
	InsertTailList(&gGlobalList, &pData->ListEntry);

	if (!IsListEmpty(&gGlobalList)
	{
		// Removes the first element from the list
		PLIST_ENTRY pListEntry = RemoveHeadList(&gGlobalList);

		// Retrieves a pointer to the beginning of the structure
		pData = CONTAINING_RECORD(pListEntry, MY_DATA, ListEntry);
	}
}
\end{lstlisting}

\section{Hash Table}

\projectname provides a basic implementation of a hash table through the HASH\_TABLE structure.
Functions to insert, lookup, remove or iterate through elements are provided, as well as a couple of
generic hashing functions for keys which have a size of less than 8 bytes.

The implementation solves collisions through chaining, thus if multiple elements hash to the same
key they will be placed in a doubly linked list. The only dynamically allocated memory needed is
when initializing the hash table, this is due to the fact that a doubly linked list is needed for
each unique key and the number of keys in a hash table is configurable and passed as a parameter
when pre-initializing the hash table.

The way we work with the hash elements is similar to the way we work with doubly linked list entries,
however instead of using LIST\_ENTRY fields we will use HASH\_ENTRY fields. In this section we will
illustrate how we can use a hash table, also, if you go to \file{hash\_table.h} you will
see some usage examples as well.

We will discus the operation in three stages: initializing the hash table, using the hash table and
destroying it.

An example of how to initialize a hash table is provided in \fullref{lst:HashInitEg}:
\begin{itemize}
	\item On lines 6-14 we have the declaration of the MY\_PROCESS structure, because it holds a
	field of type HASH\_ENTRY it can be inserted in a hash table.
	
	\item On line 23 we see a call to \func{HashTablePreInit}: here we specify the number of unique
	keys we want the hash table to hold and the size of the key. The number of unique keys must be
	known because the size allocated for the internal hash table data depends on this: for each
	unique key a doubly linked list header is required to hold the element chain. The size of the
	key is required by the hashing functions.
	
	\item On line 30 we see a call to \func{HashTableInit} which provides the newly allocated
	memory for the hash table internal structure, the hashing function and the difference in bytes
	between the field used as the key (in our case \textit{Id}) and the field used for chaining the
	element in the hash table (in our case \textit{HashEntry}).
	
	\item After the hash table has been initialized it can now be populated with elements.
\end{itemize}

\begin{lstlisting}[caption={Hash Initialization Example},label={lst:HashInitEg},numbers=left]
#define NO_OF_KEYS		8

// global hash table of processes
HASH_TABLE gHashTable;

typedef struct _MY_PROCESS
{
	PID 			Id;
	...
	// MY_PROCESS elements will link in the global list through the HashEntry field
	HASH_ENTRY		HashEntry;
	...
	BYTE			Data;
} MY_PROCESS, *PMY_PROCESS;

void InitProcessHashTable(void)
{
	// Pre-initialize the hash table, specify the maximum number of keys we want it to 
	// have and the size of the key. This function returns the size in bytes required
	// for its internal HASH_TABLE_DATA structure, we will need to allocate this
	// memory dynamically.
	DWORD requiredHashSize = HashTablePreinit(&gHashTable, NO_OF_KEYS, sizeof(PID));
	
	PHASH_TABLE_DATA pUnknown = ExAllocatePoolWithTag(0, requiredHashSize, HEAP_TEST_TAG, 0);
	ASSERT(pUnknown != NULL);
	
	// Initialize the hash table to use the HashFuncGenericIncremental hashing
	// function and specify the difference in bytes between the offset to the Key 
	// field and the offset to the HASH_ENTRY field
	HashTableInit(&gHashTable, 
		      pUnknown, 
		      HashFuncGenericIncremental, 
		      FIELD_OFFSET(MY_PROCESS, Id) - FIELD_OFFSET(MY_PROCESS, HashEntry));
}
\end{lstlisting}

The code in \fullref{lst:HashUsageEg} provides some usage examples:
\begin{itemize}
	\item On line 7 an initialized MY\_PROCESS structure is inserted into the hash table. Because
	the size of the key, the hashing function and the offset between the HASH\_ENTRY and the key
	are already known the only parameters required by this function are the hash table and the
	HASH\_ENTRY to insert.
	
	\item On line 10 the previously allocated element is removed. 
	
	\textbf{NOTE: while it is possible
	to simply perform a RemoveEntryList(\&pProcess->HashEntry) to remove the entry from the hash table
	it is not advisable to do so, the hash table maintains a count of elements and it cannot maintain
	a proper count if the explicit hash functions are not used. Also, in the future the HASH\_ENTRY
	structure may not be defined as a LIST\_ENTRY.}
	
	\item On line 15 the element whose key is 0x4 is removed from the hash table. If the element
	was not present a NULL pointer is returned.
	
	\item On line 23 a check is made to make sure that the element whose key is 0x4 is no longer
	present in the hash table.
	
	\item On line 25 the number of elements in the hash table is logged.
	
	\item On lines 29-36 an iterator is initialized and the whole hash table is traversed element by
	element. The HASH\_TABLE\_ITERATOR structure always maintains the position of the next element
	in the hash table, that's why while traversing the hash table with an iterator we can always
	remove the current element and continue the iteration. Once there are no more elements in the
	hash table \func{HashTableIteratorNext} returns NULL.
\end{itemize}

\begin{lstlisting}[caption={Hash Usage Example},label={lst:HashUsageEg},numbers=left]
void UsageFunction(void)
{
	PMY_PROCESS pProcess = ExAllocatePoolWithTag(0, sizeof(MY_PROCESS), HEAP_TEST_TAG, 0);
	// ... Initialize process structure ...

	// Insert the new element into the hash table
	HashTableInsert(&gHashTable, &pProcess->HashEntry);
	
	// Remove previously added element
	HashTableRemoveEntry(&pHashTable, &pProcess->HashEntry);
	
	// Remove a process by searching for a certain ID
	PID idToRemove = 0x4;
	
	PHASH_ENTRY pEntry = HashTableRemove(&gHashTable, &idToRemove);
	if (pEntry != NULL)
	{
		PMY_PROCESS pProcess = CONTAINING_RECORD(pEntry, MY_PROCESS, HASH_ENTRY);
		// ... work with the process structure ...
	}
	
	// Check if the element is still in the list
	ASSERT(HashTableLookup(&gHashTable, &idToRemove) == NULL);
	
	LOG("Number of elements in the hash table is \%u\n", HashTableSize(&gHashTable));
	
	// Iterate through all the elements of the list
	// The iterator maintains the current traversal state within the hash table
	HASH_TABLE_ITERATOR it;
	
	HashTableIteratorInit(&gHashTable, &it);
	
	while ((pEntry = HashTableIteratorNext(&it) != NULL)
	{
		// process the entry
	}
}
\end{lstlisting}

Finally, the code listed in \fullref{lst:HashUsageEg} provides an example of how to destroy the
hash table:
\begin{itemize}
	\item On line 19 we see the call which will empty the hash table and call the provided
	\func{ProcessFreeFunc} for each element in the hash table.
	
	\item On lines 1-10 we see the function which is called for each element removed from the hash
	table when \func{HashTableClear} is called. The Object parameter will point to the HASH\_ENTRY
	field from the structure, thus to get to the actual element we will use the CONTAINING\_RECORD
	macro. Once we have the element we can free its memory.
	
	\item On line 22 we see the call to free the memory which was allocated for the hash table
	internal implementation: it is no longer needed.
\end{itemize}

\begin{lstlisting}[caption={Hash Destruction Example},label={lst:HashDestEg},numbers=left]
void
(_cdecl ProcessFreeFunc)(
    IN      PVOID       Object,
    IN_OPT  PVOID       Context
    )
{
	PMY_PROCESS pProcess = CONTAINING_RECORD(Object, MY_PROCESS, HashEntry);
	
	ExFreePoolWithTag(pProcess, HEAP_TEST_TAG);
}

void DestroyProcessHashTable(void)
{
	PVOID pUnknown = gHashTable->TableData;

	// Free all the elements in the hash table, the ProcessFreeFunc function
	// will be called for each element, the pointer received as the first
	// parameter will point to the HASH_ENTRY field in MY_PROCESS
	HashTableClear(&gHashTable, ProcessFreeFunc, NULL);

	// Free the hash table internal memory allocated at initialization
	ExFreePoolWithTag(pUnknown, HEAP_TEST_TAG);
}
\end{lstlisting}

\section{Hardware Timers}

Modern x86 systems possess a diverse range of hardware timers which can be programmed to achieve different
functionalities:
\begin{itemize}
	\item High-precision timers: HPET and LAPIC.
	\item General usage timers: RTC, PIT, ACPI timer.
\end{itemize}

All the timers enumerated except the LAPIC timer are system-wide, i.e. there is only one hardware timer available on a
given system. The LAPIC timer is available on a per-CPU basis.

\projectname currently uses only the PIT and RTC timers. The PIT is used for dispatching the scheduler, while the RTC is
used for updating the system time. Also, code for programming the LAPIC timer is already implemented and can be used
without modification.

Because \projectname does not interract with the HPET (High Precision Event Timer) or the ACPI (Advanced Configuration
and Power Interface) timer at all these are not described here.

\subsection{PIT Timer}

The PIT (Programmable Interval Timer) has 3 programmable timers:
\begin{enumerate}
	\item Channel 0: this is used by \projectname for the scheduler interrupt. It is setup as a periodic interrupt to 
trigger every \macro{SCHEDULER\_TIMER\_INTERRUPT\_TIME\_US} $\mu$s (the default value is 40ms).

	\item Channel 1: unused.

	\item Channel 2: used by \func{PitSleep} to wait for a certain number of Microseconds to pass. This is not
programmed to generate an interrupt, it continously polls the PIT to determine if the timer has expired or not.
\end{enumerate}

The timer is initalized in \func{\_IomuSetupPit} and the interrupt registered is \func{\_IomuSystemTickInterrupt}.

Further information is available \cite{intelPch} Chapter 12.3 Timer I/O Registers and 
\href{http://wiki.osdev.org/Programmable_Interval_Timer}{OS dev - PIT}.

\subsection{RTC Timer}

\projectname uses the RTC (Real Time Clock) timer to update the system time, i.e. the one displayed in the upper right-hand
corner of the screen. It is programmed to trigger an interrupt only when the BIOS CMOS memory clock updates its second 
counter, i.e. when a second passes.

The interrupt function is \func{OsInfoTimeUpdateIsr}. It is registered in \func{\_IomuSetupRtc} - this function is also
responsible of actually programming the hardware timer to trigger only on timer updates. The RTC can also be programmed
to generate two additional interrupts:
\begin{enumerate}
	\item Periodic interrupts: the interrupt triggers each timer the timer period expires. The range of interrupt
frequencies is from 122 $\mu$s to 500 ms.

	\item Alarm interrupt: the interrupt is generated when the system time reaches a software programmed value
(hour:minute:second).
\end{enumerate}

More information about the RTC can be found in \cite{intelPch} Chapter 12.6 Real Time Clock Registers and 
\href{http://wiki.osdev.org/RTC}{OS dev - RTC}..

\subsection{LAPIC Timer}

The LAPIC (Local Advanced Programmable Interrupt Controller) timer is a hardware timer implemented on each CPU core.
It is currently not used by \projectname, but the following function is available for programming it:
\begin{lstlisting}[caption={LAPIC Timer},label={lst:LapicTimer}]
//******************************************************************************
// Function:     LapicSystemSetTimer
// Description:  Enables the LAPIC timer on the current CPU to trigger every
//               Microseconds ms. If the argument is 0 the timer is stopped.
// Parameter:    IN DWORD Microseconds - Trigger period in microseconds.
// Parameter:    OUT_PTR PTHREAD * Thread
// NOTE:         This only programs the LAPIC timer on the current CPU.
//******************************************************************************
void
LapicSystemSetTimer(
    IN      DWORD                           Microseconds
    );
\end{lstlisting}

If the argument passed is 0 the timer is stopped, else it is enabled in periodic mode to trigger every \var{Microseconds}.
When the interrupt triggers \func{\_SmpApicTimerIsr} will be executed.

More information about the LAPIC timer can be found in \cite{intelSys} Chapter 10.5.4 APIC Timer and
\href{http://wiki.osdev.org/APIC_timer}{OS dev - LAPIC Timer}.

\chapter{Debugging}
\label{chap:Debugging}

\projectname is a complex project and sometimes when you'll make a change in the code you'll see
that what's actually happening in the system differs from what you were expecting.

Unfortunately, \projectname doesn't have support for a debugger, YET! We wish to write one for the
third iteration of the project.

However, in the author's opinion sufficient tools are available to solve any bugs which may arise
in the code.

The following sections will go through the different techniques available for debugging code and
detecting errors:
\begin{itemize}
	\item Signaling function failure: by returning (preferably) unique status values for functions which could fail you could pinpoint the location of the error more easily.

	\item Logging: simply write messages logging the data which interests you, usually when there
are bugs the information logged will differ from what you were expecting.

	\item Asserts: validate all assumptions. Even if you know that the sun rises in the east or that
only threads in the dying state can be destroyed make sure those conditions stand.

	\item Disassembly: follow the assembly instructions which caused your system to crash and 
determine the call stack.

	\item Halt debugging: when you're desperate and the system reboots in an infinite loop place
some HLT instructions in the code to diagnose the problem.
\end{itemize}

\section{Signaling function failure}

When working with \projectname you'll often see that most functions return a STATUS type
value, depending on this value you can determine if the function failed or succeeded.
The recommended way of doing this is by using the \macro{SUCCEEDED} macro, which returns TRUE in
case of success, and FALSE in case of an \textbf{\color{red} error} or \textbf{\color{orange} warning} status.

The reason why this format was chosen is so that it can be used in Windows applications as well
without conflicting with existing Microsoft defined statuses. The format is defined by \href{https://docs.microsoft.com/en-us/windows-hardware/drivers/kernel/defining-new-ntstatus-values}{Microsoft} - you don't need to know the details, but the idea is that this format is very extensible and can be used by many different vendors (or in our case components) without overlaying results.

\subsection{Interpreting STATUS values}

The easiest way of interpreting a STATUS value is by copying its value in the Windows
calculator application - or any other application which displays the bit values of a number.

The STATUS type is simply a DWORD - 4 bytes and is interpreted in the following
way:

0x{\color{blue}S}{\color{purple}CCC}{\color{green}VVVV}.

Where

\begin{itemize}
	\item S stands for Severity:
		\begin{enumerate}
			\item 0xE - Error
			\item 0xA - Warning
			\item 0x4 - Informational
			\item 0x0 - Success
		\end{enumerate}

	\item CCC stands for Component:
		\begin{enumerate}
			\item 0x800 - General
			\item 0x100 - CPU
			\item 0x080 - Communication
			\item 0x040 - Timers
			\item 0x020 - Heap
			\item 0x010 - Memory
			\item 0x008 - Storage
			\item 0x004 - Disk
			\item 0x002 - Apic
			\item 0x001 - Device
		\end{enumerate}
		
	\item VVVV stands for the value within the component - unfortunately there are too many values
	to enumerate here, but once you go to \file{status.h} you can easily determine the exact status.
\end{itemize}

Let's take an example of a status value and determine what it means: we have the following
log message: \textit{[ata.c][82]AtaInitialize failed with status: 0xE0010001}.
\begin{itemize}
	\item First of all, the text between the first set of brackets tells us the file in which the
message was logged - ata.c
	\item Then, we have the number between the second set of brackets which indicate the line number
- 81.
	\item And finally we see the status value 0x{\color{blue}E}{\color{purple}001}{\color{green}0001}.

	We always start interpreting from the LEFT to the RIGHT (starting from the MSB going to the LSB).

	The severity is 0xE meaning it is an error message.
	The component is 0x001 meaning it is a device error message.
	And finally the value of the general error message is 0x0001, going to \file{status.h} and
searching for the value within the statuses marked with \macro{DEVICE\_MASK} we find the status
name being \macro{STATUS\_DEVICE\_DOES\_NOT\_EXIST}.
\end{itemize}

When the status name is not explicit enough to understand what it means or when it is returned you
can always search in the whole solution to see where that status value is used.

Ideally, statuses should not be too generic, but they should be specific enough for the programmer
to be able to pinpoint the location (or few locations) from which the value could have been returned.

\textbf{NOTE: It is recommended that you add new status values when you're working on your project
and the code added can fail in a way not described by the existing values.}

\section{Logging}

The easiest thing to do if you don't understand something is to log it. There are different logging
levels which you may use depending on the importance of the logged information: trace, info, warning
and error.

You can setup the logging level you want to use when calling the \func{LogSystemInit} for
initializing the logging system or \func{LogSetLevel} for changing the log level any time.

For a log message to be shown, its log level must be greater than or equal to the logging level set,
 i.e. if currently the system logging level is LogLevelWarning then only warning and error messages
will be displayed and info and trace messages will be ignored.

Also, there are different components which may log trace messages, and you may activate/deactivate
logging trace messages based on the component logging them. This can be set, as logging level, 
either when initializing the logging system or by calling \func{LogSetTracedComponents} at a later
time. For example, if you want to log all trace messages only from the generic and exception
component you would either do as shown in \fullref{lst:LoggingInit} or as shown in
\fullref{lst:LoggingChange} after logging has been initialized.

\begin{lstlisting}[caption={Logging Init},label={lst:LoggingInit}]
    LogSystemInit(LogLevelTrace,
                  LogComponentGeneric | LogComponentException,
                  TRUE
                  );
\end{lstlisting}

\begin{lstlisting}[caption={Logging Change},label={lst:LoggingChange}]
	LogSetTracedComponents(LogComponentGeneric | LogComponentException);
\end{lstlisting}

All of the logging functions and the definitions can be found in the \file{log.h} file. At first,
things may seem confusing, but here's the basic things that you need to know:

\begin{itemize}
	\item \func{LOG} If you want to log a simple message.

	\item \func{LOGP} If you want to log a message plus the CPU id from which it is logged.

	\item \func{LOGL} If you want to log a message plus the file and line form which the log
occurs.

	\item \func{LOGTPL} If you want to log a message plus the thread, CPU, file and line from which
the log occurs.

	\item \func{LOG\_WARNING} If you want to log warnings, these should be unexepected
things from which the function can recover.

	\item \func{LOG\_ERROR} If you want to log errors, these should be used if the errors cannot be
handled and the function emitting the message failed execution.

	\item \func{LOG\_TRACE\_THREAD} If you want to log a message on behalf of the threading
component and include the CPU id and file and line from which the log occurred.
\end{itemize}

Other logging mechanism exist for each component: \func{LOG\_TRACE\_*} functions. Be careful when
logging, not to log too much in functions called often, this can slow down the system considerably
until the point that close to 0 progress is made.

Logging is safe to be used in both normal executing code and in interrupt handlers because it uses
primitive locks for synchronization. However, you cannot log messages in the logging functions, this
would cause infinite recursion and the OS will crash due to a \#PF caused by a stack overflow which
cannot be solved.

If you want to dump a raw memory region to find out what's there you can use the \func{DumpMemory}
function or you can use more specialized functions available in the \file{dmp\_*.h} files which
display information about a specific component/device/entity in a more organized way. As an example
see \func{DumpInterruptStack} or \func{DumpProcess}.

\section{Asserts}

Another mechanism to make sure everything works as expected is to use asserts. The code is already
full of them (1000+ instances). \textbf{\color{red}DO NOT REMOVE ANY OF THE EXISTING ASSERTS!}

When you place an assert in the code you set as the condition the thing you're expecting to be true
a.k.a an invariant, if the condition does not hold once execution reaches that point the current CPU
 will stop execution, notify the other CPUs that a fatal error has occurred and log the condition
which failed the assert.

As an example, lets go to the \func{\_ThreadSchedule} function and have a look at one of its asserts
: ASSERT(INTR\_OFF == CpuIntrGetState()). When execution will reach this point the condition will
be verified to see if it's true, i.e. interrupts are disabled when entering the function, if this is
not the case and for some reason the interrupts are enabled execution will stop and you will see the
following message in the log file:

\begin{verbatim}
[ERROR][hal_assert.c][29][CPU:00]Kernel panic!
[ERROR][hal_assert.c][31][CPU:00]Thread: [main-00]
[ERROR][hal_assert.c][33][CPU:00][ASSERT][thread.c][1029]Condition: ((0 == 
CpuIntrGetState())) failed
\end{verbatim}

One can easily see that 0 == CpuIntrGetState() didn't hold true and you can see that the check is
happening in the \file{thread.c} file at line 1029.

\projectname is full of such asserts to make sure the functions are called correctly and if some
code changes all the invariants are still respected. This project is a large one and the work effort
invested in it spans almost a year and without these asserts it's very easy to forget how a code
change in a component can affect other components and alter the system's behavior. This is why
asserts are used and this is why you should not remove any of them.

Each time you work on your project you should be asking yourself, what conditions have to hold for
the function to work properly? These conditions should validate the state of the system and, if the
function is one which can be only be used by other OS components, the parameters. You should NOT
assert the validity of parameters received in a system call because these are user provided
parameters, but you should assert if your \func{ThreadSetPriority} function receives an invalid
priority because this function can only be called by other TRUSTED OS components.

To continue the example when writing the \func{ThreadSetPriority} function you should assert that
the thread priority is a valid one and that \func{GetCurrentThread} returns a valid non-NULL thread.

\section{Dissassembly}

Sometimes, when you make code changes some errors may occur which are not caught by asserts or any
other type of validation. These errors may lead to exceptions on the processor which will cause the
system to crash.

In case such an event occurs \projectname will log the interrupt stack and the processor registers
when the exception took place. The only way to see where the operation took place is to disassemble
the instructions near the RIP which caused the exception. We recommend using IDA \cite{ida} for this.

Most of the time, it is easy to pinpoint the function and the exact place where the exception took
place once we look at the disassembled code. Either because there is a log function close to the
faulting RIP which also logs the file and line or we can easily determine the function to which the
instruction belongs to.

We will now look into an example on how to diagnose a \#PF using the IDA dissasembler.

We made some changes to the code and now \projectname crashes due to an un-handled page fault. The
last lines of the log file looks like this:
\begin{verbatim}
[thread.c][184][CPU:00]_ThreadInit succeeded
[ERROR][isr.c][149]Could not handle exception 0xE [#PF - Page-Fault Exception]

Interrupt stack:
Error code: 0x0
RIP: 0xFFFF800001032CFB
CS: 0x18
RFLAGS: 0x10086
RSP: 0xFFFF85014365E850
SS: 0x20

Control registers:
CR0: 0x80010031
CR2: 0x20
CR3: 0x1140000
CR4: 0x100020
CR8: 0x0

Processor State:
RAX: 0x0
RCX: 0xC0000100
RDX: 0xFFFF850100000000
RBX: 0x80800
RSP: 0xCCCCCCCCCCCCCCCC
RBP: 0xFFFF800001006125
RSI: 0x408
RDI: 0xFFFF85014365E860
R8: 0xFFFF85014365E8B8
R9: 0xFFFF85014365E658
R10: 0xFFFF800001118000
R11: 0x1
R12: 0x0
R13: 0x0
R14: 0x0
R15: 0x0
RIP: 0xCCCCCCCCCCCCCCCC
Rflags: 0xCCCCCCCCCCCCCCCC
Faulting stack data:
[0xFFFF85014365E850]: 0xCCCCCCCCCCCCCCCC
[0xFFFF85014365E858]: 0xCCCCCCCCCCCCCCCC
[0xFFFF85014365E860]: 0xFFFF85014365E918
[0xFFFF85014365E868]: 0xFFFF80000103330F
[0xFFFF85014365E870]: 0x0
[0xFFFF85014365E878]: 0xCCCCCCCCCCCCCCCC
[0xFFFF85014365E880]: 0xCCCCCCCCCCCCCCCC
[0xFFFF85014365E888]: 0xCCCCCCCCCCCCCCCC
[0xFFFF85014365E890]: 0xCCCCCCCCCCCCCCCC
[0xFFFF85014365E898]: 0xCCCCCCCCCCCCCCCC
[0xFFFF85014365E8A0]: 0xCCCCCCCCCCCCCCCC
[0xFFFF85014365E8A8]: 0xCCCCCCCCCCCCCCCC
[0xFFFF85014365E8B0]: 0xCCCCCCCCCCCCCCCC
[0xFFFF85014365E8B8]: 0x80800000306C3
[0xFFFF85014365E8C0]: 0x1FABFBFFF7FA3223
[0xFFFF85014365E8C8]: 0xCCCCCCCCCCCCCCCC
[0xFFFF85014365E8D0]: 0xCCCCCCCCCCCCCCCC
[0xFFFF85014365E8D8]: 0xCCCCCCCCCCCCCCCC
[0xFFFF85014365E8E0]: 0xCCCCCCCCCCCCCC00
[0xFFFF85014365E8E8]: 0xCCCCCCCCCCCCCCCC
[0xFFFF85014365E8F0]: 0xCCCCCCCCCCCCCCCC
[0xFFFF85014365E8F8]: 0xCCCCCCCCCCCCCCCC
[0xFFFF85014365E900]: 0x4110860067F20473
[0xFFFF85014365E908]: 0xCCCCCCCCCCCCCCCC
[0xFFFF85014365E910]: 0xCCCCCCCCCCCCCCCC
[0xFFFF85014365E918]: 0xFFFF85014365EBB8
[0xFFFF85014365E920]: 0x80800
[0xFFFF85014365E928]: 0xFFFF80000106AE90
[0xFFFF85014365E930]: 0x0
[0xFFFF85014365E938]: 0xFFFF850140000390
[0xFFFF85014365E940]: 0xFFFF8000010EECA8
[0xFFFF85014365E948]: 0xFFFF8000010EEC9C
[ERROR][hal_assert.c][29][CPU:00]Kernel panic!
[ERROR][hal_assert.c][31][CPU:00]Thread: [main-00]
[ERROR][hal_assert.c][33][CPU:00][ASSERT][isr.c][152]Condition: (exceptionHandled)
failed
Exception 0xE was not handled
\end{verbatim}

We can clearly see the value of the faulting RIP as being 0xFFFF800001032CFB and the faulting address
0x20 (in CR2). When you see these low valued addresses you can be sure that they are caused by a
NULL pointer dereference, i.e. a field at offset 0x20 from a structure is accessed through a NULL
pointer.

We can now open the binary file with IDA and dissasemble the code.

\textbf{NOTE: the .pdb file should be in the same folder as the binary file when you open it with
 IDA. Because of this, our recommendation is to load the \file{HAL9000.bin} file from the bin folder.}

\textbf{NOTE: Be careful not have re-compiled \projectname since the time of the crash.
If you have done so, the disassembled instructions will not match those which generated the crash.}

Once we opened the binary in IDA we can use the Jump to Address (G keyboard shortcut) to jump to the 
faulting instruction: 0xFFFF800001032CFB. In our case we see \fullref{fig:IdaFault}.

\begin{figure}
	\centering
	\includegraphics[scale=0.7]{IdaFaulting}
		\caption{IDA disassembly near RIP}
	\label{fig:IdaFault}
\end{figure}

\begin{figure}
	\centering
	\includegraphics[scale=0.7]{IdaReferences}
		\caption{\func{ProcessGetName} References}
	\label{fig:IdaRefs}
\end{figure}

As previously stated, RAX+0x20 is dereferenced and because RAX is NULL the faulting address is 0x20.
The function in which this takes place is \func{ProcessGetName}. If we press CTRL + X once the
function name is selected we can see that this function is only called from 3 other functions -
illustrated in \fullref{fig:IdaRefs}.

From these 3 functions we can deduce that the one which actually called \func{ProcessGetName} is
\func{ProcessThreadInsertIntoList} because the last thing logged before the exception is from
\file{thread.c} line 184 which belongs to the \func{ThreadSystemInitMainForCurrentCPU} function,
which doesn't interact with the other 2 candidate functions.

Usually, at this point we can remember exactly (or look at the diffs in our source control system)
to see what what code changes we have made in this region.

In case this is not enough information we can also look at the stack dump from the end of the log to
determine the call stack hierarchy. Addresses which begin with 0xFFFF85 correspond to dynamically
allocated memory, so we only have to look at addresses which are of the form 0xFFFF8000010 These
addresses belong either to code, or to data segments. This can be determined with 100\% precision
by looking at the PE, but because there are very few such addresses on the stack we can easily use
Jump to Address (G) to see the position of each address.

If we go to  0xFFFF80000103330F we will see that we're in \func{ProcessInsertThreadInList}, if we
want to go further up the hierarchy we look for the next potential code address candidate. Once we
find it, in our case 0xFFFF80000106AE90, we go to it and discover that the address belongs to
\func{ThreadSystemInitMainForCurrentCPU}.

The next candidates are  0xFFFF8000010EECA8 and  0xFFFF8000010EEC9C, however once we go to their
addresses we will see that these are data values and not code. If we want more data dumped from the
stack we can increase the value of \macro{STACK\_BYTES\_TO\_DUMP\_ON\_EXCEPTION} macro from
\file{isr.c}, recompile \projectname and re-run the code.

\section{Halt debugging}

Sometimes you will have bugs in the early boot stages of the operating system. Unfortunately, until
the interrupt handlers are setup every exception generated by the processor leads to a system reboot.
This is because each exception generated leads to a double fault (\#DF) because the interrupt
handlers are not setup and there's nothing to handle the exception. The double fault can't be
handled either and a triple fault is generated, a triple fault automatically reboots the system.

\projectname initializes the interrupt handlers in \func{InitIdtHandlers} which is called by
\func{SystemInit} pretty early in the initialization stage (before the memory managers are
initialized).

It is possible, due to bugs in the code, for the processor to generate triple faults in some
situations even after the interrupt handlers have been setup. However this is very unlikely, if you
have a triple fault it is very probable it is caused by the code executed before the call to
\func{InitIdtHandlers}.

If the problem occurs before the call to \func{SerialCommunicationInitialize} it is impossible for
you to log anything. The method I suggest for debugging these issues is by placing HLT instructions
causing the processor's execution to stop.

Pick a place in \projectname, insert a call to \func{\_\_halt}, recompile and run. There are two
likely possibilities:
\begin{itemize}
	\item The machine hangs without rebooting. If this happens, you know that the HLT instruction
was reached and executed. That means that whatever caused the reboot must be after the place you
inserted the HLT. Now move HLT later in the code sequence.

	\item The machine reboots in a loop. If this happens, you know that the machine didn't make it
to the HLT instruction. Thus, whatever caused the reboot must be before the place you inserted the
HLT instruction. Now move the HLT instruction earlier in the code sequence.
\end{itemize}

If you move around the HLT instruction in a "binary search" fashion, you can use this technique to
pin down the exact spot that everything goes wrong. It should only take a few minutes at most.

\textbf{NOTE: Instead of using the HLT instruction you could place an infinite loop as suggested in
the Pintos documentation.}

\chapter{Development Tools}

\section{Setting Up the Environment for \projectname}
\label{sect:SetupBuild}

The following steps describe the setup required for configuring your development system to be
able to run, build and test \projectname. Except the first 2 steps, and the last one all the other
steps can be done automatically by running the install script \file{HAL9000\_install.pl}
(see \fullref{sect:AutoConf}).
\begin{enumerate}
	\item Download and install 
\href{https://www.visualstudio.com/post-download-vs/?sku=community&clcid=0x409&downloadrename=true}{Visual Studio 2015 Community}.

	\item Download and install
\href{http://www.activestate.com/activeperl/downloads/thank-you?dl=http://downloads.activestate.com/ActivePerl/releases/5.24.0.2400/ActivePerl-5.24.0.2400-MSWin32-x86-64int-300560.exe}{Active Perl x86}.

	\item Download and install \href{http://www.vmware.com/go/tryworkstation}{VMWare workstation}
(\textbf{VMWare player will not good or other virtualization solutions are not good!}). You can
download it from here and later activate it using your student license.

	\item Unzip \projectname's source code, we will refer to the folder where you unzipped it as
\var{PROJECT\_ROOT\_DIRECTORY}.

	\item Unzip the VMWare Virtual Disk Development Kit, we will refer to the folder where you
unzipped the files as \var{PATH\_TO\_VM\_TOOLS}.

	\item Unzip the VMWare Vix archive, we will refer to the folder where you unzipped the files as
\var{PATH\_TO\_VIX\_TOOLS}.

	\item Unzip the two virtual machines HAL9000\_VM and PXE\_VM. We will refer to the folder where
you unzipped the HAL9000\_VM as \var{PATH\_TO\_HAL9000\_VM}.

	\item Unzip the PXE archive, we will refer to the folder where you unzipped the files as
\var{PATH\_TO\_PXE}.

	\item Open the HAL9000 VM and change its processor configuration so that the number of virtual
CPUs given to the machine equals to the number of CPUs available on the physical machine.

	\item Edit the \var{PROJECT\_ROOT\_DIRECTORY}/src/postbuild/paths.cmd file by adding the
following lines before the ":end" label:

\begin{verbatim}
:config_\var{COMPUTER_NAME}

set PXE_PATH=\var{PATH_TO_PXE}
set PATH_TO_VM_DISK=\var{PATH_TO_HAL9000_VM}\HAL9000.vmdk
set PATH_TO_VM_TOOLS=\var{PATH_TO_VM_TOOLS}
set VOL_MOUNT_LETTER=Q:
set PATH_TO_VIX_TOOLS=\var{PATH_TO_VIX_TOOLS}
set PATH_TO_LOG_FILE=\var{PATH_TO_HAL9000_VM}\HAL9000.log
set PATH_TO_VM_FILE=\var{PATH_TO_HAL9000_VM}\HAL9000.vmx

goto end
\end{verbatim}

And the following line after "if \_\%COMPUTERNAME\%\_==\_ALEX-PC\_ goto config\_ALEX-PC" :
\begin{verbatim}
if _%COMPUTERNAME%_==_\var{COMPUTER_NAME}_ goto config_\var{COMPUTER_NAME}
\end{verbatim}

Where the \textbackslash raw variables must be replaced with the proper paths as mentioned in the
configuration steps.

	\item Create a new virtual network as described in \fullref{sect:VirtNetwork}.
\end{enumerate}

\subsection{Automatic Configuration}
\label{sect:AutoConf}

Because of the many steps involved in the configuration of the project, we wrote a perl installion
script which handles all the configuration steps except the first two and the last one.

Unfortunately we haven't been able to automate those steps so you'll have to do them manually.

Once you completed the first two steps you should download each archive described in the
configuration step and place them all in the same folder as the \file{HAL9000\_install.pl} perl
script. The folder layout should look like \fullref{fig:AutoFolder} or \fullref{fig:AutoFolderTC}.

\begin{figure}
	\centering
	\includegraphics[scale=0.76]{AutoFolder}
		\caption{Folder structure for HAL9000\_install.pl}
	\label{fig:AutoFolder}
\end{figure}

\begin{figure}
	\centering
	\includegraphics[scale=0.7]{AutoFolderTC}
		\caption{Folder structure for HAL9000\_install.pl}
	\label{fig:AutoFolderTC}
\end{figure}

You can run the script without any parameters and the script will extract all the archives in a
HAL9000 directory which it will create in the current directory. The script will do all the
configuration steps described in steps 3 to 10 (inclusive).

Once the script finishes execution, all you are left with is the last step, i.e. configuring the
virtual network, a guide is available at \fullref{sect:VirtNetwork}.

\subsection{Virtual Network Creation}
\label{sect:VirtNetwork}

A host-only virtual network must be configured before to be able to perform a PXE boot. The
following steps must be followed:

\begin{enumerate}
	\item Open VMWare workstation and in the \textit{Edit} menu choose the \textit{Virtual Network
Editor} option as shown in \fullref{fig:VMwareMenu}.

\begin{figure}
	\centering
	\includegraphics[scale=0.5]{VMWare_VirtNetMenu}
		\caption{Edit -> Virtual Network Editor}
	\label{fig:VMwareMenu}
\end{figure}

	\item Once in the \textit{Virtual Network Editor} click \textit{Change Settings}.

	\item Click \textit{Add Network...} and select \textit{VMnet1} as the network to add. See
\fullref{fig:VmWareAddNetwork}.

\begin{figure}
	\centering
	\includegraphics[scale=0.75]{VMWare_AddVirtNet}
		\caption{Add new network}
	\label{fig:VmWareAddNetwork}
\end{figure}

	\item Once you added a new network select it as configure it as a \textit{Host-onl}y network,
check the \textit{Connect a host virtual adapter to this network} option and un-check the
\textit{Use local DHCP service to distribute IP address to VMs} option. Write in the
\textit{Subnet IP} field 192.168.224.0 and set the \textit{Subnet mask} to 255.255.255.0. This is
illustrated in \fullref{fig:VmWareVirtNetConf}

\begin{figure}
	\centering
	\includegraphics[scale=0.75]{VMWare_VirtNetConf}
		\caption{VMnet1 Configuration}
	\label{fig:VmWareVirtNetConf}
\end{figure}

\end{enumerate}

\subsection{System Architecture}

The final architecture after you have successfully configured your system should look like the one
illustrated in \fullref{fig:SystemArch}.

You will have a virtual network called \textit{VMnet1} with network IP 192.168.224.0, subnet
255.255.255.0 and DHCP on host disabled.

You will have a PXE VM connected on VMnet1 which will host a DHCP and TFTP server and has a static
IP assigned to 192.168.224.2. This VM has access to a shared folder from the host operating system
from which it will supply the boot image to the HAL9000 VM for network boot through PXE.

You will have a HAL9000 VM which is connected to VMnet1 and which will boot from the network. On
boot PXE VM will provide the boot image and the OS will be automatically loaded and started. All
the messages logged by HAL9000 VM will be written into the \file{HAL9000.log} file on the host
machine.

You will have a Visual Studio solution which will hold many projects, 2 of which are very important:
\begin{itemize}
	\item The HAL9000 project: upon successful build the binary is copied to the PXE shared folder
so the PXE VM will be able to instantly provide the new image to the HAL9000 VM.

	\item RunAllTests project: copies the Tests.module file to the PXE share and starts the HAL9000
VM. The .module file is the one which tells the HAL9000 operating system which tests it needs to run
and validate.

	After the HAL9000 VM has finished execution the log file, \file{HAL9000.log}, will be parsed by
the project and the results of each test and a summary will be displayed.
\end{itemize}

\begin{figure}
	\centering
	\includegraphics[scale=0.96]{VMArchitecture}
		\caption{System Architecture}
	\label{fig:SystemArch}
\end{figure}

\subsection{Troubleshooting}

\subsubsection{HAL9000 VM not receiving an IP}

If you're in the situation illustrated in \fullref{fig:DHCPFail} it means you are not receiving an
IP address. The reason is one of the following:
\begin{itemize}
	\item VMnet1 is not configured properly. It is important to use the IP and subnet addresses
specified in \fullref{sect:VirtNetwork}.

	\item HAL9000 VM is not connected to VMnet1. You can check this by going to the virtual machine,
\textit{Edit virtual machine settings} -> \textit{Hardware} -> \textit{Network Adapter} and make
sure the network selected is Custom: VMnet1. Also make sure \textit{Connect at power on} is checked
for the network adapter.

	\item PXE VM is not connected to VMnet1. Apply the same steps as if HAL9000 VM was not connected
to VMnet1.

	\item PXE VM is not turned on.
\end{itemize}

\begin{figure}
	\centering
	\includegraphics[scale=0.86]{DHCPFail}
		\caption{DHCP fail}
	\label{fig:DHCPFail}
\end{figure}

\subsubsection{HAL9000 VM TFTP open timeout}

If you're in the situation illustrated in \fullref{fig:TFTPFail} it means the TFTP folder is not
accessible to the PXE VM.

This is probably caused by the fact that the shared folders are disabled for the PXE VM. To fix the
problem power off the PXE VM and go to \textit{Edit virtual machine settings} -> \textit{Options} ->
\textit{Shared Folders} and make sure \textit{Always Enabled} is checked. Re-start the PXE VM and the
problem should be solved.

\begin{figure}
	\centering
	\includegraphics[scale=0.86]{TFTPFail}
		\caption{TFTP fail}
	\label{fig:TFTPFail}
\end{figure}

\subsubsection{HAL9000 VM boot file not found}

If you're in the situation illustrated in \fullref{fig:BootFileNotFound} it means that the shared
PXE folder is missing the HAL9000.bin file. The following reasons are most probable:
\begin{itemize}
	\item HAL9000 project compilation failed. Make sure you successfully compiled your project.
Rebuild the HAL9000 project to be sure.

	\item You may have skipped some configuration steps and because of this the HAL9000 project does
not copy the output file to the PXE folder. Take a look in the PXE folder and see if a file appears
after you successfully compiled HAL9000, if it does not go back to \fullref{sect:SetupBuild} and
try to figure out what configuration step you've skipped.
\end{itemize}

\begin{figure}
	\centering
	\includegraphics[scale=0.86]{BootFileNotFound}
		\caption{File not found}
	\label{fig:BootFileNotFound}
\end{figure}

\section{Visual Studio}
\label{sect:VisualStudio}

Before starting work in visual studio (VS) we \textbf{RECOMMEND} that you install the Visual Assist
\cite{visualAssist} extension. It is a very handy tool which enhances your IDE experience and helps
you write better and faster code.

Now, that you've setup your environment as described in \fullref{sect:SetupBuild} we can now get down
to business. To open the project in VS you need to go to the HAL9000/\file{src} folder and open the
\file{HAL9000.sln} file.

This will open the \projectname VS project and you will see something like \fullref{fig:VSSolution}
in front of you. This is because the solution contains many more smaller projects, some of these
projects are the HAL9000 project (the core itself), the hardware abstraction layer, drivers, user
applications and other utilities. You can see a basic description of the projects in
\fullref{sect:SourceTree}.

\begin{figure}
	\centering
	\includegraphics[height=1.3\textwidth]{VS_Solution}
		\caption{\projectname Solution}
	\label{fig:VSSolution}
\end{figure}

Each project is in turn divided into filters, for example in the HAL9000 project we can see the
following filters: \textit{Header Files}, \textit{Source Files} and we can go deeper in the hierarchy.
For example, we can go through the \textit{Source Files -> devices -> apic} filter to see the files
responsible for managing the IOAPIC and the LAPIC. At first it may seem cumbersome to navigate the
project this way, but the files are organized logically (by component) and once you learn where each
functionality is it will be easy for you to navigate through the project.

Other tips for easy navigation can be found in \fullref{sect:KeyboardShortcuts}.

\subsection{Keyboard Shortcuts}
\label{sect:KeyboardShortcuts}

\begin{itemize}
	\item By pressing "CTRL + ," a search bar appears on your top right and you will be able to
search symbols (functions and variables), data definitions and files dynamically while typing. This
searches through external projects and dependencies as well.

	\item By pressing "ALT + SHIFT + S" a find symbol box opens and you can search only for symbols.
\textbf{NOTE: This is only available with Visual Assist}.

	\item By pressing "ALT + SHIFT + O" you can type the file name to which you want to jump to.
\textbf{NOTE: This is only available with Visual Assist}.

	\item By pressing "ALT + SHIFT + F" while a symbol, data structure or definition is selected
a smart search will occur and all its references will be displayed. \textbf{NOTE: This is only
available with Visual Assist}.

	\item By pressing "ALT + SHIFT + R" while a symbol, data structure or definition is selected
you can change its name and all the instances will reflect its new name - smart search and replace.
\textbf{NOTE: This is only available with Visual Assist}.

	\item By pressing "ALT + M" while in a file a drop-down will appear listing all the symbols
defined in this file. \textbf{NOTE: This is only available with Visual Assist}.

	\item By Pressing "ALT + G" (with Visual Assist) or "F12" on a function you can either go to
its declaration or its implementation. \textbf{NOTE: The Visual Assist option usually works better}.

	\item By pressing "CTRL + G" while in a file you can jump to any line.

	\item By pressing "ALT + O" while in a file you can toggle between the .h and the .c file. For
example if you're in the thread.c file and press ALT + O you will be in the thread.h file, press it
again and you'll return to the c file.

	\item By pressing "CTRL + SHIFT + F" you can launch a text search in the whole solution, project
or files whose names follow a certain pattern.

	\item By pressing "CTRL + F" you can perform a text search the current file.

\end{itemize}

\section{Hg}

It's crucial that you use a source code control system to manage your \projectname code. This will
allow you to keep track of your changes and coordinate changes made by different people in the 
project. For this class we recommend that you use hg \cite{tortoiseHg}. If you don’t
already know how to use hg, we recommend that you read the guide at \href{http://hginit.com/}{hginit}.

For hosting your project we recommend that you use Bitbucket \cite{bitbucket}. You can use any 
hosting site you like as long as the project repository is a private one, i.e. it is not visible to
any users outside your team (and maybe your TA).

\chapter{Coding Style}

Unfortunately, due to the fact that the work done on \projectname spans a long period (almost a
year) some minor coding style changes have been made and the code is not 100\% consistent with the
coding style.

\section{Functional rules}
\begin{itemize}
	\item Global variable usage should be as limited as possible, instead use static file variables,
this restricts access to the variable only to the functions in the current compilation unit (c file).

	Keeping track of global variables can be very hard and it is very poor design to allow for a
variable to be modified by any piece of code. If data must be modified by more components special
functions should exist to do this, see \func{LogSetState} and friends, \func{ThreadExecuteForEachThreadEntry}
and similar functions which can be called from any OS module, and modify static file variables.

	\item When a function should only be used by a single C file it should be a static function.

	\item Instead of using goto cleanup construct, use the \_\_try \_\_finally construction, see
\href{https://msdn.microsoft.com/en-us/library/9xtt5hxz.aspx}{MSDN try-finally} for details and
\func{ProcessCreate} for a usage example.

	\item Functions which can be called from outside the trust boundary (between different projects
or privilege levels) should validate all the parameters.

	\item Internal functions which can be called only from inside the trust boundary (inside the
same project) should NOT validate any arguments, however it is a good technique to use ASSERTs to
validate the function's parameters.

	\item Validate the successful execution of any function you're calling that returns a status.

	\item Functions should be annotated using SAL.
\end{itemize}

\section{Non-functional Rules}
\begin{itemize}
	\item Function names should be UpperCamelCase: \func{MmuPreinitSystem}, \func{ThreadCreate}.
	\item Local variables names should be lowerCamelCase: status, firstArg, secondArg.
	\item Structure field names should be UpperCamelCase: SystemUptime, TickCountEarly, AllList.

	\item Static functions names should be \_UpperCamelCase: \func{\_ThreadReference}.
	\item Static local variables names should be \_\_lowerCamelCase: \_\_currentEntry;
	\item Static file variables names should be: m\_lowerCamelCase: m\_coreData;

	\item The names of variables which hold pointers should be preceded by a p: pFirstArg,
\_\_pCurrentEntry, m\_pData;
	\item The names of variables which hold BOOLEAN values should be preceded by a b: bFound,
\_\_bFinished, m\_bInitialized;

	\item Lines should not be longer than 120 columns.

	\item Use SPACES instead of TABS.

	Tab size and indent size should be set to 4. You can set this in Visual Studio by accessing \textit{Tools -> Options -> Text Editor -> C/C++ -> Tabs} and checking the \textit{Insert spaces} radio button.

	\item Do not leave empty lines or whitespaces at the end of the line.

	We recommend installing the \textit{Trailing Whitespace Visualizer} plugin for Visual Studio. It can be installed by accessing \textit{Tools -> Extensions and Updates -> Online}. It is a free plugin and everytime you save a file it automatically removes trailing whitespaces and empty lines.

\end{itemize}


\end{appendices}