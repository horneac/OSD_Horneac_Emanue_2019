\chapter{Introduction}
\label{chap:Introduction}

\projectname is a 64-bit operating system (OS) written for x86 symmetric multiprocessor systems
(SMP). It provides a simple round robin scheduler for scheduling threads on the available CPUs,
support for launching user-mode applications, a basic FAT32 file-system driver and basic drivers for
providing I/O operations (VGA, IDE disk, keyboard, legacy COM and ethernet).

\projectname can theoretically run on any physical Intel x86 PC which supports 64-bit operating 
mode. However, because it is cheaper, easier and safer to test in a virtual environment we will use
VMWare Workstation \cite{vmware} for running our OS.

The project is a semester long and consists of improving the threading and user-mode support
currently available in \projectname.

This chapter provides the basic introduction to the project, highlights the differences between this
project and a very popular one (Pintos \cite{pintos}) which served as an inspiration for the 
conception of \projectname. Afterward, this section provides the basic pointers to navigating,
building, running and testing the code.

\section{Versus Pintos}

\projectname started as the author's project during his first graduate year and as it evolved the
author started thinking it could maybe replace Pintos  as the OS used for teaching
students operating system concepts.

The arguments made in favor of using \projectname instead of Pintos are the following:
\begin{itemize}
	\item Support for multi-processor systems. This provides true concurrency and makes
synchronization a bit harder (you can't just disable interrupts) and provides an environment more
authentic to the real world. For more information on these differences see \fullref{sect:MultiCore}.

	\item 64-bit execution mode. Several benefits are gained by the OS by
running in 64 bit mode (we will refer to this as long mode from now on): some of these benefits are
performance wise: PCIDs, syscall/sysret instruction pair while some enhance security: XD, larger
virtual address spaces and protection keys. For more details, see \fullref{sect:Why64}.

	\item The OS image is multiboot compliant \cite{gnuMultiboot}, this means it can be loaded by
any multiboot loader such as grub or PXE. This makes it easy to run the OS through both hard-disk
boot and network boot.

	\item Code is annotated using an annotation language to provide better compile-time checks and
detect errors earlier. Besides improving code quality by detecting checks earlier, annotating code
teaches students to better think of their design before starting to write functions by forcing them
to annotate the function header before one line of the body is written. See \cite{msdnSAL} for more
 information about the annotation language used.

	\item If people are not interested strictly in OS design, but want to write their own file
system, network or disk driver they could easily extend this OS to support any file system, network
adapter or disk controller interface.

\end{itemize}

The arguments made in favor of using Pintos instead of \projectname are:

\begin{itemize}
	\item No debugger support - we have plans to change this in the future. Unfortunately, because
gdb doesn't know how to interpret pdb files it isn't as simple as adding a basic gdb stub for
debugger support. \textbf{NOTE: This is on the feature list for the next project iteration.}

	\item Pintos is tried and tested, being part of the curriculum for many top ranked universities
in the world for many years while \projectname is at its first iteration.

	\item Pintos doesn't rely as many CPU features and doesn't require the host operating system to
be a 64-bit version.


\end{itemize}

Before going any further, we wish to acknowledge Pintos's contribution both as source code
inspiration and as project requirements definition. Some parts of the Pintos documentation which
apply to \projectname are taken as is without modification and some are altered.

\section{Project Overview}

\subsection{Root Directory}

The root directory as generated using the install script (see \fullref{sect:SetupBuild}) should look
like the one illustrated in \fullref{fig:ProjectRoot}.

\begin{figure}
\begin{verbatim}
.
|-- [ dir]  bin
|-- [ dir]  acpi
|-- [ dir]  docs
|-- [ dir]  postbuild
|-- [ dir]  PXE
|-- [ dir]  src
|   |-- [ dir]  Ata
|   |-- [ dir]  CommonLib
|   |-- [ dir]  CommonLibUnitTests
|   |-- [ dir]  Disk
|   |-- [ dir]  Eth_82574L
|   |-- [ dir]  FAT32
|   |-- [ dir]  HAL
|   |-- [ dir]  HAL9000
|   |-- [file]  HAL9000.sln
|   |-- [file]  HAL9000_WithoutApplications.sln
|   |-- [ dir]  NetworkPort
|   |-- [ dir]  NetworkStack
|   |-- [ dir]  PE_Parser
|   |-- [ dir]  shared
|   |-- [ dir]  SwapFS
|   |-- [ dir]  Usermode
|   |-- [ dir]  Utils
|   |-- [ dir]  Volume
|-- [ dir]  temp
|-- [ dir]  tests
|-- [ dir]  tools
|-- [ dir]  VM
9 directories, 0 files
\end{verbatim}
\caption{Project Root Directory}
\label{fig:ProjectRoot}
\end{figure}

\begin{itemize}
	\item bin: this is the place where the binaries are placed after compilation.
	\item acpi: contains the external acpica library. It is used for parsing the ACPI tables.
	\item docs: contains this documentation and documentation related to the hardware components or
external libraries used by \projectname.
	\item postbuild: contains the scripts which write the OS binary to the PXE folder, write
the user applications to the VM hard disk and run the tests. You will not run these scripts directly, they are part of VS projects.
	\item src:
		\begin{itemize}
			\item HAL9000.sln: project solution - this is what you need to open with VS.
			\item HAL9000\_WithoutApplications.sln same as HAL9000.sln except it does not contain
		the user-mode projects. If you feel that VS is working too slow with the full project try
		using this.
			\item Usermode: contains all the user-mode applications and the common user-mode library.
			\item Other folders belong to their respective projects, you will only have to work with
		the HAL9000 project, but we suggest you navigate the project only in VS, see
		\fullref{sect:SourceTree}.
		\end{itemize}
	\item temp: required by VS for storing temporary files - ignore it but do not delete the folder.
	\item tests: contains the code for parsing the test results and for validating them.
	\item tools: contains miscellaneous tools including the assembler and the tools required to
	interact with the VM and its disk.
	\item VM: contains the two VMs: the one which will boot \projectname and a PXE server.
	\item PXE: contains the contents required by the PXE server - this folder is shared with the
	PXE server VM. Upon successful compilation of \projectname its binary is placed here for network
	boot. \textbf{DO NOT TOUCH THIS FOLDER.}
\end{itemize}

\subsection{Source Tree Overview}
\label{sect:SourceTree}

We highly recommend using the Visual Studio (VS) as an IDE and not using notepad++ or other editors
for navigating the source code. The whole source tree is based on VS filter for grouping files, i.e.
you will only see a logical separation between files if you open VS - on the file system all the
files are in the same folder regardless of the component they belong to.

To open the project in VS you need to open the \file{HAL9000.sln} file from the src directory. You
will see several projects loaded in the VS solution:
\begin{itemize}
	\item FAT32

	Provides a simple implementation of a FAT32 file system driver. You will not work here.

	\item SwapFS

	Responsible for managing the swap partition : provides simple read/write functionality at a 4KB
granularity. You will not work here.

	\item PE\_parser

	Parses portable executable (PE) files, i.e. Windows executables and synthesizes
information for use by other components. The module is required for re-mapping the kernel and for
loading user-mode applications into memory. You will not work here.

	\item Eth\_82574L

	Driver for the Intel 82574 GbE Controller Family. The network card emulated by VMWare belongs to
this family and the driver implements all the required functionality for network reception and
transmission. You will not work here.

	\item NetworkPort

	Provides helper functions for ethernet drivers. This layer is added so that it is not required
of each ethernet driver (such as Eth\_82574L) to implement the same functionality over and over
again. This layer provides helper functions which can be used by any ethernet driver. You will not
work here.

	\item NetworkStack

	Provides an interface for the operating system to access the network devices. You will not work
here.

	\item Ata

	Provides the IDE disk controller driver, it is responsible for performing disk I/O by
communicating with the hard disk controller. You will not work here.

	\item Disk, Volume

	These drivers provide abstraction for the operating system and the attached file systems. A file
system will be mounted on a volume, while a volume will belong to a disk and the disk will be
on top of a hard-disk controller. You will not work here.

	NOTE: In real systems the hierarchy can be more complex (a file-system may be on multiple
volumes and a volume may be on multiple disks, but in \projectname these mappings are one to one).

	\item RemoveAllTests

	When this project is built causes \projectname's next boot not to run any tests and to accept
user-given commands through the keyboard.

	\item RunTests

	Will start a VM instance of the operating system and will wait until \projectname finishes
running all the tests or until a timeout occurs (default: 5 minutes). Once execution of the VM
finishes the results of the tests will be compared with the expected results and a summary of the
results will be displayed. More details in \fullref{sect:Testing}.

	\item CommonLib

	Contains some basic data structures and functions which would normally be present in the C
standard library and several other generic constructs which are not coupled with \projectname. Some
of the features include: string manipulation functions, reference counters, bitmaps, spinlocks,
hash tables, support for termination handlers and so on.

	\item CommonLibUnitTests

	Not relevant to the project or any of the laboratories. If you want to write user-mode C++ code
to test the \textit{CommonLib} functionality here's the place where you can do it.

	\item HAL

	Provides the layer which works directly with the x86 architecture hardware components (CMOS,
RTC, IOAPIC, LAPIC, PCI) and processor structures (CPUID, MSR, GDT, IDT, MTRR, TSS).

	\item HAL9000

	The actual operating system code. \textbf{This is the place where all your work will take place}.
The project contains too many files and too much functionality to describe it all here. On each
project you will get a more detailed viewed of the components you will work on. You can also browse
the code at any time to determine what each component does.

\end{itemize}

For more information on how to navigate the solution you can read \fullref{sect:VisualStudio}.

\subsection{Building \projectname}

Before building the project in VS you will first need to configure the \file{paths.cmd} file - this
is explained in \fullref{sect:SetupBuild}.

Once you have finished configuring the \file{paths.cmd} file you can now build the solution. If you
want to simply build the OS without starting the VM and running all the tests it is enough to build
the HAL9000 project.

If you want to go through the whole test cycle rebuild the RunTests project. See more details in
\fullref{sect:Testing}.

If you want to make sure next time you boot the OS no test will be run rebuild the RemoveAllTests
project.

\subsection{Running \projectname}

Firstly, you should have both your VMs configured as described in \fullref{sect:SetupBuild}. We will
refer to the VM on which the PXE server resides as the PXE VM and the VM on which we'll run
\projectname as the \projectname VM.

You have to start up the PXE VM first and wait for it to boot, you do not have to login because the
PXE server automatically starts on system boot. Once the first VM is started started, the second VM
(on which our OS will run) can be launched. Due to its BIOS settings this VM will boot from the
network and the PXE server will hand it the \projectname binary as the boot image.

Once the PXE VM is up and running you can start the HAL9000 VM by hand from the VMWare Workstation
interface, or you can build the RunAllTests project to validate your current implementation of one
of the projects.

If you just want to give some commands by hand once the OS is up you can build the RemoveAllTests
project and then start the HAL9000 VM from the VMWare Workstation interface.

\section{Grading}

We will grade your assignments based on the design quality and test results, each of which comprises
50\% of your grade.

\subsection{Design}

We will judge your design based on the design document. We will read your entire design document.
Don’t forget that design quality, including the design document, is 50\% of your project grade. 
It is better to spend one or two hours writing a good design document than it is to spend that time
getting the last 5\% of the points for tests.

\subsubsection{Design Document}

We provide a design document template for each project. For each significant part of a
project, the template asks questions in four areas:

\begin{itemize}
	\item \textbf{Data Structures}

Copy here the declaration of each new or changed struct or struct member, global or static variable,
typedef, or enumeration. Identify the purpose of each in 25 words or less.

The first part is mechanical. Just copy new or modified declarations into the
design document, to highlight for us the actual changes to data structures. Each
declaration should include the comment that should accompany it in the source
code (see below).

We also ask for a very brief description of the purpose of each new or changed
data structure. The limit of 25 words or less is a guideline intended to save
your time and avoid duplication with later areas.

	\item \textbf{Algorithms}

This is where you tell us how your code works, through questions that probe your understanding of 
your code. We might not be able to easily figure it out from the code, because many creative 
solutions exist for most OS problems. Help us out a little.

Your answers should be at a level below the high level description of requirements given in the 
assignment. We have read the assignment too, so it is unnecessary to repeat or rephrase what is 
stated there. On the other hand, your answers should be at a level above the low level of the code
itself. Don’t give a line-by-line run-down of what your code does. Instead, use your answers to 
explain how your code works to implement the requirements.

	\item \textbf{Synchronization}

An operating system kernel is a complex, multithreaded program, in which
synchronizing multiple threads can be difficult. This section asks about how
you chose to synchronize this particular type of activity.

	\item \textbf{Rationale}

Whereas the other sections primarily ask “what” and “how,” the rationale section concentrates on 
“why.” This is where we ask you to justify some design decisions, by explaining why the choices you 
made are better than alternatives. You may be able to state these in terms of time and space 
complexity, which can be made as rough or informal arguments (formal language or proofs are
unnecessary).

An incomplete, evasive, or non-responsive design document or one that strays from the template 
without good reason may be penalized. See \fullref{chap:ProjDoc} for a sample design document for a
fictitious project.

\end{itemize}

\subsection{Testing}
\label{sect:Testing}

Your test result grade will be based on our tests. Each project has several tests, testing the
implementation you provide. To completely test your submission build the RunTests project and wait
for the results to appear (may take up to 5 minutes). The following are performed when RunTests is
built:
\begin{enumerate}
	\item Parses the contents of the threads or userprog directory from within the tests folder to
determine which tests must be run, for every test files found in these directories a test will be
run.

	\item Generates the \file{Tests.module} file and copies it to the PXE location - this file
contains all the commands that must be executed by the OS to run all the tests previously parsed
and to shutdown.

	\item Starts the \projectname VM and waits for its termination.

	\item Waits for the VM to terminate or, if it runs for more than 5 minutes, it forcefully shuts
it down.

	\item The serial log is parsed and divided into .result files having the names of the tests
executed. As an example: if the OS ran the TestsThreadStart a TestsThreadStart.result file will be
placed by the already existent TestsThreadStart.test file.

	\item Each .result file is then parsed and a conclusion is drawn by following these steps:
	
	\begin{enumerate}
		\item If an error is logged during execution the test is marked as FAIL.
		\item If a PASS string is logged during execution the test is marked as PASS.
		\item If a .check file is present then the perl script within it is run to validate the
	.result file.
		\item If none of the previous applied, the  .result file is compared with the corresponding
		.test file expecting an identical file for passing the test.
	\end{enumerate}

	\item An .outcome file is generated for each test containing the test conclusion (PASS or FAIL)
and the lines which caused the conclusion - the reason may be an error logged or a PASS text or the
lines differing between the .result and .test files.

	\item The result of each test is shown, a summary of the results for each category and the total
pass / total count are displayed.
\end{enumerate}

For further details, you can check the \file{execute\_tests.pl} file in the tests folder. If you
want to run only a single test you can run the \file{run\_single\_test.cmd} script found in the
src/postbuild folder. The parameters taken by the script are the project name and the test category\textbackslash name.
As an example, if you would want to run the "TestThreadTimerAbsolute" test you need to run the
the following command: \textit{run\_single\_test.cmd threads timer\textbackslash TestthreadTimerAbsolute}.

Each test provides feedback only at completion, not during its run, to see the reason why a test
succeeded or failed see the .outcome file for the test. As an example: if your implementation failed
the ThreadPriorityMutex test you can open the ThreadPriorityMutex.outcome file in the
tests/threads/priority-scheduler folder to determine the cause of the FAIL.

All of the tests and related files are in the tests directory. Before we test your submission,
we will replace the contents of that directory by a pristine, unmodified copy, to ensure that the
correct tests are used. Thus, you can modify some of the tests if that helps in debugging, but we
will run the originals.

All software has bugs, so some of our tests may be flawed. If you think a test failure is a bug in
the test, not a bug in your code, please point it out. We will look at it and fix it if necessary.

Please don’t try to take advantage of our generosity in giving out our test suite. Your code has to
work properly in the general case, not just for the test cases we supply. For example, it would be
unacceptable to explicitly base the kernel’s behavior on the name of the running test case. Such
attempts to side-step the test cases will receive no credit. If you think your solution may be in a
gray area here, please ask us about it.

\section{Useful documents and links}

\begin{itemize}
	\item \cite{osdev} - Hobbyist OS development wiki and forum. Contains a lot of tutorials on a large range of topics
starting from low-level device programming to high level concepts such as scheduling. There is also a forum available
where you can find additional information on certain topics or ask questions.

	\item \cite{osdever} - Contains a tutorial for developing an operating system from scratch. Some interesting topics
include memory management, programming interrupts, implementing spinlocks and developing a GUI.

	\item \cite{intelSys} - Intel System Manual: the definitive documentation for everything related to the Intel CPU:
topics which may help you for the projects are Chapters 4 (Paging), 6 (Interrupt and Exception Handling), 8 (Multiple-Processor
Management).

	\item \cite{intelInstr} - Intel Instruction Set Reference: if you are interested in the exact effects of an x86
assembly instruction this is the manual for you.

\end{itemize}